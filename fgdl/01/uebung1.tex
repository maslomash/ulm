\documentclass{article}
\usepackage{amsmath}
\usepackage[utf8]{inputenc}
\usepackage[T1]{fontenc}
\usepackage[ngerman]{babel}
\usepackage[shortlabels]{enumitem}
\usepackage{amsfonts}

\title{Formale Grundlagen: Übung 1}
\author{Alexander Waldenmaier, Tutorin: Vanessa Hiebeler}

\begin{document}
    \maketitle
    \subsection*{Aufgabe 1.1}
    Hinweis: Ich verstehe es so, dass die Menge einer Zahl nicht das selbe wie die Zahl selbst ist, also
    $x \neq \{x\}$, aber $x \in \{x\}$.
    \begin{enumerate}[(a)]
        \item $M_a = \{3\}$
        \item $M_b = \{2,4,5,6\}$
        \item $M_c = \{1, \{3\}\}$
        \item $M_d = \{\{1\}, \{2\}, \{1, 2\}\}\Delta\{1, \{1, 2\}, \emptyset\} = \{\{1\}, \{2\}, 1\}$
        \item $M_e = \{7, 14, 21, 28, 35, 42\}$
        \item $M_f = \{\{3, 4, 5\}, \{4, 3, 5\}, \{6, 8, 10\}, \{8, 6, 10\} \}$
    \end{enumerate}


    \subsection*{Aufgabe 1.2}
    \begin{enumerate}[(a)]
        \item 
        \begin{align*}
            |M_a| &= \binom{|\{a, b, c, d\}|}{3} = \frac{|\{a, b, c, d\}|!}{3!(|\{a, b, c, d\}-3)!} = \frac{4!}{3!(4-3)!} = \frac{24}{6*1}=4
        \end{align*}
        \item 
        \begin{align*}
            |M_b| =& |\{\{1\}, \{3\}, \{4\}, \{1,3\}, \{1,4\}, \{3,4\}, \{1,3,4\}\}\Delta ... \\
            &\{\{1\}, \{2\}, \{4\}, \{1,2\}, \{1,4\}, \{2,4\}, \{1,2,4\}\}| \\
            =& |\{\{2\}, \{3\}, \{1, 2\}, \{1,3\}, \{2,4\}, \{3,4\}, \{1,2,4\}, \{1,3,4\}\}| = 8
        \end{align*}
        \item $|M_c| = |\emptyset| = 0$
        \item $|M_d| = |\emptyset| = 0$
    \end{enumerate}


    \subsection*{Aufgabe 1.3}
    \begin{enumerate}[(a)]
        \item Bijektiv, da für alle Elemente des Wertebereichs $y \in Y$ genau ein Element aus dem Definitionsbereich 
        $x \in X$ existiert. (Frage: Müsste hier nicht korrekterweise der Wertebereich alle natürlichen Zahlen UND Null enthalten?)
        \item Nicht injektiv, da beispielsweise $(x, y) = (1, -1)$ das selbe Ergebnis liefert wie $(x,y) = (-1, 1)$. Surjektiv, 
        da für jedes Element des Wertebereichs mindestens eine Wertekombination des Definitionsbereichs existiert, für den
        $f_b(x,y) = \frac{x}{y}$ gilt (Grund: Definition der rationalen Zahlenmenge). Da nicht sowohl injektiv als auch surjektiv, ist
        die Funktion nicht bijektiv. 
        \item Nicht injektiv, da beispielsweise $x=1$ und $x=-1$ beide das Ergebnis $f_c(x) = 1$ liefern. Surjektiv, da alle
        natürlichen Zahlen genau die Menge aller positiver ganzen Zahlen ist, was durch die Betragsfunktion dargstellt wird. Da nicht sowohl injektiv als auch surjektiv, ist
        die Funktion nicht bijektiv. 
        \item Nicht surjektiv, da beispielsweise für die Zahl $y=0.5$ aus dem Wertebereich keine zugehörige Zahl aus dem Definitionsbereich
        gefunden werden kann, die $f_d(x) = x^2$ erfüllt. Injektiv, da für jede Zahl aus dem Wertebereich entweder genau eine Zahl aus dem
        Definitionsbereich vorliegt, die die Funktionsvorschrift erfüllt, oder keine. Ersteres tritt für alle Zahlen auf, die
        Quadratzahlen sind, da in diesem Fall ausschließlich $x=|\sqrt{y}|$ möglich ist ($x=-|\sqrt{y}|$ liegt bspw. nicht
        im Definitionsbereich). Zweiteres tritt für alle Zahlen auf, die keine Quadratzahlen sind. Da nicht sowohl injektiv als auch surjektiv, ist
        die Funktion nicht bijektiv. 
    \end{enumerate}


    \subsection*{Aufgabe 1.4}
    Zu beweisen:
    \begin{align*}
        \sum^n_{i=1}(-1)^i=-1
    \end{align*}
    Wenn $n$ ungerade ist, so lässt es sich darstellen als $i=2k+1$ wobei $k \in \mathbb{N}_0$. Eingesetzt in die obige Gleichung
    ergibt das:
    \begin{align*}
        A(k): \sum^{2k+1}_{i=1}(-1)^i=-1
    \end{align*}
    Beweis durch vollständige Induktion:

    \paragraph*{Induktionsanfang:} $A(k_0), k_0=0$
    \begin{align*}
        \sum^{1}_{i=1}(-1)^i= (-1)^1 = -1
    \end{align*}
    
    \paragraph*{Induktionsschritt:} $A(k) \rightarrow A(k+1)$ \\\\
    Induktionsbehauptung: $\sum^{2k+1}_{i=1}(-1)^i=-1$
    \begin{align*}
        A(k+1): \sum^{2(k+1)+1}_{i=1}(-1)^i &= -1\\
        \sum^{2k+3)+1}_{i=1}(-1)^i &= -1 \\
        \sum^{2k+1}_{i=1}(-1)^i + (-1)^{2k+2} + (-1)^{2k+3} &= -1 \\
        \sum^{2k+1}_{i=1}(-1)^i + 1 - 1 &= -1\\
        \sum^{2k+1}_{i=1}(-1)^i &= -1
    \end{align*}
    Die letzte Zeile entspricht wieder der Induktionsbehauptung, weshalb der Induktionsschritt bewiesen wurde. \\\\
    \textbf{q.e.d.}


    \subsection*{Aufgabe 1.5}
    Zu beweisen:
    \begin{align*}
        A(n, k): \frac{n!}{k!^2} \in \mathbb{N} | n, k \in \mathbb{N} \land n = 2k
    \end{align*}
    \paragraph*{Direkter Beweis:}
    \begin{align*}
        A(n, k): \frac{n!}{k!^2} = \frac{n!}{k!k!} = \frac{n!}{k!(2k-k)!} = \frac{n!}{k!(n-k)!} = \binom{n}{k} \in \mathbb{N}
    \end{align*}
    Aus der Definition des Binomialkoeffizienten ergibt sich die Aussage. \\\\
    \textbf{q.e.d.}

\end{document}