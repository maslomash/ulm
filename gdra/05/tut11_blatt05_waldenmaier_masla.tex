\documentclass{article}
\usepackage{amsmath}
\usepackage[utf8]{inputenc}
\usepackage[T1]{fontenc}
\usepackage[ngerman]{babel}
\usepackage{amsfonts}
\usepackage[left=3cm,right=2cm,top=2.5cm,bottom=2cm]{geometry}
\usepackage{karnaugh-map}

\title{Grundlagen der Rechnerarchitektur: Übungsblatt 5}
\author{Alexander Waldenmaier, Maryia Masla}

\begin{document}
    \maketitle
	\subsection*{Aufgabe 1: Minimierung zu Ehren Maurice Karnaugh}
	Die Wahrheitstabelle von $f(x_1,x_2,x_3)$ und $g(x_1,x_2,x_3)$:
	\begin{table*}[h]
		\centering
		\begin{tabular}{c|c|c|c|c}
			$x_3$ & $x_2$ & $x_1$ & $f(x_1,x_2,x_3)$ & $g(x_1,x_2,x_3)$ \\ \hline
			0     & 0     & 0     & 1                & 0                \\
			0     & 0     & 1     & 0                & 1                \\
			0     & 1     & 0     & 0                & 0                \\
			0     & 1     & 1     & 1                & 0                \\
			1     & 0     & 0     & 0                & 1                \\
			1     & 0     & 1     & 1                & 1                \\
			1     & 1     & 0     & 1                & 0                \\
			1     & 1     & 1     & 1                & 1
		\end{tabular}
	\end{table*}
	\begin{itemize}
		\item[a)] Die DKNF und KKNF lassen sich für beide Funktionen aus der Wahrheitstabelle ablesen.\\\\
		DKNF:
		\begin{align*}
			&f(x_1,x_2,x_3)=\overline{x_3x_2x_1} + \overline{x_3}x_2x_1 + x_3\overline{x_2}x_1 + x_3x_2\overline{x_1} + x_3x_2x_1 \\
			&g(x_1,x_2,x_3)=\overline{x_3x_2}x_1 + x_3\overline{x_2x_1} + x_3\overline{x_2}x_1 + x_3x_2x_1
		\end{align*}
		KKNF:
		\begin{align*}
			&f(x_1,x_2,x_3)=(x_3+ x_2+\overline{x_1}) \cdot (x_3 +\overline{x_2}+ x_1) \cdot (\overline{x_3}+ x_2 + x_1) \\
			&g(x_1,x_2,x_3)=(x_3 + x_2 + x_1) \cdot (x_3 + \overline{x_2} + x_1) \cdot (x_3 + \overline{x_2} + \overline{x_1}) \cdot (\overline{x_3} + \overline{x_2} + x_1)
		\end{align*}\\
		\item[b)]
		\begin{minipage}[t]{0.45\textwidth}
			Vollständig minimierte DKNF von f(): \\\\
			$f(x_1,x_2,x_3)=\overline{x_3x_2x_1} + x_3x_2 + x_3x_1 + x_2x_1$
		\end{minipage}
		\begin{minipage}[h]{0.5\textwidth}
			\begin{karnaugh-map}[4][2][1][$x_3x_2$][$x_1$]
				\minterms{0,7,3,6,5}
				\autoterms[0]
				\implicant{5}{7}
				\implicant{7}{6}
				\implicant{3}{7}
				\implicant{0}{0}
			\end{karnaugh-map}
		\end{minipage}
		\item[c)] 
		\begin{minipage}[t]{0.45\textwidth}
			Vollständig minimierte KKNF von g(): \\\\
			$g(x_1,x_2,x_3)=(x_3 + \overline{x_2}) \cdot (x_3 + x_1) \cdot (\overline{x_2} + x_1)$
		\end{minipage}
		\begin{minipage}[h]{0.5\textwidth}
			\begin{karnaugh-map}[4][2][1][$x_3x_2$][$x_1$]
				\maxterms{0,1,3,5}
				\autoterms[1]
				\implicant{0}{1}
				\implicant{1}{3}
				\implicant{1}{5}
			\end{karnaugh-map}
		\end{minipage}
	\end{itemize}
	
	
    \subsection*{Aufgabe 2: Moment - Warum eigentlich minimieren?}
    \begin{itemize}
    	\item[a)]Die Wertetabelle der Funktion $f(x_1,x_2,x_3,x_4)$:
		\begin{table*}[h]
			\centering
    		\begin{tabular}{c|c|c|c|c|c}
    			$x_1$ & $x_2$ & $x_3$ & $x_4$ & $(x_1, x_2, x_3, x_4)_{10}$ & $f(x_1,x_2,x_3,x_4)$ \\ \hline
    			0   &   0   &   0   &   0   &  0 &        1           \\
    			0   &   0   &   0   &   1   &  1 &          0           \\
    			0   &   0   &   1   &   0   &  2 &          1           \\
    			0   &   0   &   1   &   1   &  3 &          1           \\
    			0   &   1   &   0   &   0   &  4 &          1           \\
    			0   &   1   &   0   &   1   &  5 &          1           \\
    			0   &   1   &   1   &   0   &  6 &          1           \\
    			0   &   1   &   1   &   1   &  7 &          0           \\
    			1   &   0   &   0   &   0   &  8 &          1           \\
    			1   &   0   &   0   &   1   &  9 &          1           \\
    			1   &   0   &   1   &   0   & 10 &          1           \\
    			1   &   0   &   1   &   1   & 11 &          0           \\
    			1   &   1   &   0   &   0   & 12 &          1           \\
    			1   &   1   &   0   &   1   & 13 &          1           \\
    			1   &   1   &   1   &   0   & 14 &          1           \\
    			1   &   1   &   1   &   1   & 15 &          1
    		\end{tabular}
		\end{table*}\\
		mit
    	\begin{equation*}
    		f(x_1,..,x_4) =
    		\begin{cases}
    			1 &\text{falls } (x_1x_2x_3x_4)_{10} \mod 4 = 1 \\
    			1 &\text{falls die Quersumme von } (x_1x_2x_3x_4)_{10} = 6 \text{ ist} \\
    			1 &\text{falls } (x_1x_2x_3x_4)_{10} \mod 2 = 0 \\
    			1 &\text{falls } (x_1x_2x_3x_4)_{10} = 3 \\
    			0 &\text{sonst}
    		\end{cases}
    	\end{equation*}
    	\item[b)] DKNF:
    	\begin{align*}
    		f(x_1,x_2,x_3,x_4) &= \overline{x_1x_2x_3x_4} + \overline{x_1x_2}x_3\overline{x_4} + \overline{x_1x_2}x_3x_4 + \overline{x_1}x_2\overline{x_3x_4} + \overline{x_1}x_2\overline{x_3}x_4 + \overline{x_1}x_2x_3\overline{x_4} + x_1\overline{x_2x_3x_4} \\
    		& + x_1\overline{x_2x_3}x_4 + x_1\overline{x_2}x_3\overline{x_4} + x_1x_2\overline{x_3x_4} + x_1 x_2 \overline{x_3} x_4 + x_1x_2x_3\overline{x_4} + x_1x_2x_3x_4
    	\end{align*}
    	\item[c)] KKNF:
    	\begin{align*}
    		f(x_1,x_2,x_3,x_4) &= (x_1 + x_2 + x_3 + \overline{x_4}) \cdot (x_1 + \overline{x_2} + \overline{x_3} + \overline{x_4}) \cdot (\overline{x_1} + x_2 + \overline{x_3} + \overline{x_4})
    	\end{align*}\newpage
    	\item[d)] DNF und KNF:\newline
		\begin{minipage}[h]{0.6\textwidth}
			\vspace{-3cm}
    		\begin{align*}
    			f(x_1,x_2,x_3,x_4) &= \overline{x_1} \overline{x_2} x_3 + x_2 \overline{x_3} + x_1 \overline{x_3} + x_1 x_2 + \overline{x_4} \\\\
				&= (x_1 + x_2 + x_3 + \overline{x_4}) \cdot (x_1 \overline{x_2} + \overline{x_3} + \overline{x_4}) \\
				&\;\;\cdot (\overline{x_1} + x_2 + \overline{x_3} + \overline{x_4})
    		\end{align*}
    	\end{minipage}
    	\begin{minipage}[h]{0.3\textwidth}
			\begin{karnaugh-map}[4][4][1][$x_3x_4$][$x_1x_2$]
				\minterms{0,2,3,4,5,6,8,9,10,12,13,14,15}
				\maxterms{1,7,11}
				\implicant{3}{2}
				\implicant{4}{13}
				\implicant{12}{14}
				\implicant{12}{9}
				\implicantedge{0}{8}{2}{10}	
			\end{karnaugh-map}
    	\end{minipage}
    	\item[e)]Aus dem KV-Diagramm von $f$ lässt sich ablesen, dass die KKNF nicht weiter minimiert werden kann d.h. die Maxterme sind gleichzeitig Primimplikanten. Die DKNF kann von 13 Mintermen auf 4 Primimplikanten reduziert werden. Außerdem beinhalten Primimplikanten weniger Literale: vorher $13\cdot 4=52$, nach Minimierung $10$.\\\\
    	Die Minimierung von (Schalt-) Funktionen bringt in Hinsicht auf elektrische Schaltungen folgende Vorteile:
    	\begin{itemize}
    		\item Reduzierung vom benötigten Platz/Raum (kleinere Funktion $\rightarrow$ weniger Verknüpfungen/Operationen $\rightarrow$ kleinere Größe der Schaltung $C(S)$)
    		\item Laufzeitoptimierung (bei Minimierung der Funktion eventuell kleinere Tiefe der Schaltung $D(S)$)
    		\item Kostenreduzierung durch ersparte Materialien und Energie 
    	\end{itemize}
    \end{itemize}	


	\subsection*{Aufgabe 3: A B C-MOS}
	\begin{enumerate}
		\item[a)] Der Ausgang $f$ ist direkt mit der Spannung $V_{cc}$ verbunden, wodurch der zunächst eine logische 1 wiedergibt. Eine logische 0 kann dort nur entstehen, wenn an dem Knoten zwischen $f$ und der Spannung eine Verbindung zur Masse hergestellt wird. Daraus lässt sich leicht eine KNF ablesen: 
		\begin{align*}
			f_N(x_1, x_2, x_3) &= \overline{(x_1 x_2 + x_2 x_3 + \overline{x_1} \overline{x_2} x_3)} \\
			&= (\overline{x_1} + \overline{x_2}) \cdot (\overline{x_2} + \overline{x_3}) \cdot (x_1 + x_2 + \overline{x_3}) \\
			&= (\overline{x_1} + \overline{x_2}) \cdot (x_1 + \overline{x_3})
		\end{align*} 
		\item[b)] Wertetabelle:
		\begin{table*}[h]
			\centering
			\begin{tabular}{ccc|c}
				$x_1$ & $x_2$ & $x_3 $ & $f(x_1, x_2, x_3)$ \\ \hline
				0 & 0 & 0 & 1 \\
				0 & 0 & 1 & 0 \\
				0 & 1 & 0 & 1 \\
				0 & 1 & 1 & 0 \\
				1 & 0 & 0 & 1 \\
				1 & 0 & 1 & 1 \\
				1 & 1 & 0 & 0 \\
				1 & 1 & 1 & 0
			\end{tabular}
		\end{table*}
		\item[c)] Um von der NMOS-Schaltung zur PMOS-Schaltung zu gelangen, invertieren wir zunächst die Funktion $f_N$:
		\begin{align*}
			f_P(x_1, x_2, x_3) &= \overline{f_N(x_1, x_2, x_3)} = \overline{(\overline{x_1} + x_2) \cdot (x_1 + \overline{x_3})} \\
			&= \overline{(\overline{x_1} + x_2)} + \overline{(x_1 + \overline{x_3})} \\
			&= x_1 x_2 + \overline{x_1} x_3
		\end{align*}
		
	\end{enumerate}




\end{document}