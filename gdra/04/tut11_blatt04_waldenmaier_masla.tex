\documentclass{article}
\usepackage[utf8]{inputenc}
\usepackage[T1]{fontenc}
\usepackage[ngerman]{babel}
\usepackage[shortlabels]{enumitem}
\usepackage{amsmath}
\usepackage{amsfonts}
\usepackage[left=3cm,right=2cm,top=2.5cm,bottom=2cm]{geometry}
\usepackage{xcolor}
\usepackage{mathtools}
\usepackage{karnaugh-map}

\title{Grundlagen der Rechnerarchitektur: Übungsblatt 4}
\author{Maryia Masla, Alexander Waldenmaier}

\def\orr{\text{ OR }}
\def\andd{\text{ AND }}
\def\nor{\text{ NOR }}
\def\nand{\text{ NAND }}
\newcommand{\nyet}{\overline}

\begin{document}
    \maketitle

	\subsection*{Aufgabe 1: Teileralgebra}
	Alle Elemente von $T$ sind Teiler der 30: $30 = 2\cdot 15 = 3 \cdot 10 = 5 \cdot 6 = 1 \cdot 2 \cdot 3 \cdot 5$. Für alle Elemente $a \in T$, gilt offensichtlich: $30>a$ und $a$ ist ein Teiler von 30. Daraus folgt: 
	\begin{itemize}
		\item Neutralelement: $ggT(a, 30) = a$
		\item Absorption: $kgV(ggT(a, 30), 30) \stackrel{\substack{\text{Neutral-} \\ \text{element}}}{=} kgV(a, 30) = 30$
	\end{itemize}
	

	\subsection*{Aufgabe 2: De-morgansche Gesetze}
	\begin{enumerate}
		\item[a)] Um $\overline{x_1 + x_2} = \overline{x_1} \cdot \overline{x_2}$ zu beweisen, nutzen wir Komplementarität und zeigen, dass $\overline{x_1} \cdot \overline{x_2} \cdot (x_1 + x_2) = 0$ und $\overline{x_1} \cdot \overline{x_2} + (x_1 + x_2) = 1$ gilt:\\\\
		\begin{minipage}[t]{0.27\textwidth}
			\textbf{Wertetabelle}:\\\\
			\begin{tabular}[t]{cc|cc}
				$x_1$ & $x_2$ & $\nyet{x_1 + x_2}$ & $ \nyet{x_1} \cdot \nyet{x_2}$ \\ \hline
				0 & 0 & 1 & 1 \\
				0 & 1 & 0 & 0 \\
				1 & 0 & 0 & 0 \\
				1 & 1 & 0 & 0
			\end{tabular}
		\end{minipage}
		\begin{minipage}[t]{0.35\textwidth}
			\textbf{Algebraischer Beweis}: 
			\begin{align*}
				0 &\stackrel{!}{=} \nyet{x_1} \nyet{x_2} \cdot \nyet{\nyet{x_1 + x_2}} \\
				&\stackrel{\text{P7}}{=} \nyet{x_1} \nyet{x_2} \cdot (x_1 + x_2) \\
				&\stackrel{\text{P4}}{=} \nyet{x_1} \nyet{x_2} x_1 + \nyet{x_1} \nyet{x_2} x_2 \\
				&\stackrel{\text{P9, P6}}{=} 0 + 0 \\
				&\stackrel{\text{P5'}}{=} 0 \quad \checkmark
			\end{align*}
		\end{minipage}
		\begin{minipage}[t]{0.3\textwidth}
			\vspace{0.3em}
			\begin{align*}
				1 &\stackrel{!}{=} \nyet{x_1} \nyet{x_2} + \nyet{\nyet{x_1 + x_2}} \\
				&\stackrel{\text{P7}}{=} \nyet{x_1} \nyet{x_2} + x_1 + x_2 \\
				&\stackrel{\text{P4'}}{=} (x_1 + x_2 + \nyet{x_1})(x_1 + x_2 + \nyet{x_2}) \\
				&\stackrel{\text{P9'}}{=} (x_2 + 1)(x_1 + 1) \\
				&\stackrel{\text{P6'}}{=} 1\cdot 1 \\
				&\stackrel{\text{P3}}{=} 1 \quad \checkmark
			\end{align*}
		\end{minipage}\\\\
		\item[b)] $\overline{x_1 \cdot x_2} = \overline{x_1} + \overline{x_2}$ gilt, wenn $(\overline{x_1} + \overline{x_2}) \cdot x_1 \cdot x_2 = 0$ und $(\overline{x_1} + \overline{x_2}) + x_1 \cdot x_2 = 1$ gilt (Komplementarität): \\\\
		\begin{minipage}[t]{0.27\textwidth}
			\textbf{Wertetabelle}:\\\\
			\begin{tabular}[t]{cc|cc}
				$x_1$ & $x_2$ & $\nyet{x_1 \cdot x_2}$ & $ \nyet{x_1} + \nyet{x_2}$ \\ \hline
				0 & 0 & 1 & 1 \\
				0 & 1 & 1 & 1 \\
				1 & 0 & 1 & 1 \\
				1 & 1 & 0 & 0
			\end{tabular}
		\end{minipage}
		\begin{minipage}[t]{0.35\textwidth}
			\textbf{Algebraischer Beweis}:
			\begin{align*}
				0 &\stackrel{!}{=} (\nyet{x_1} + \nyet{x_2}) \cdot \nyet{\nyet{x_1 x_2}}  \\
				&\stackrel{\text{P7}}{=} (\nyet{x_1} + \nyet{x_2}) \cdot x_1 x_2 \\
				&\stackrel{\text{P4}}{=} \nyet{x_1} x_1 x_2 + \nyet{x_2} x_1 x_2 \\
				&\stackrel{\text{P9, P6}}{=} 0 + 0 \\
				&\stackrel{\text{P5'}}{=} 0 \quad \checkmark
			\end{align*}
		\end{minipage}
		\begin{minipage}[t]{0.3\textwidth}
			\vspace{0.3em}
			\begin{align*}
				1 &\stackrel{!}{=} \nyet{x_1} + \nyet{x_2} + \nyet{\nyet{x_1 x_2}} \\
				&\stackrel{\text{P7}}{=} \nyet{x_1} + \nyet{x_2} + x_1 x_2 \\
				&\stackrel{\text{P4'}}{=} (\nyet{x_1} + \nyet{x_2} + x_1)(\nyet{x_1} + \nyet{x_2} + x_2) \\
				&\stackrel{\text{P9'}}{=} (\nyet{x_2} + 1)(\nyet{x_1} + 1) \\
				&\stackrel{\text{P6'}}{=} 1\cdot 1 \\
				&\stackrel{\text{P3}}{=} 1 \quad \checkmark
			\end{align*}
		\end{minipage}
	\end{enumerate}


	\subsection*{Aufgabe 3: Äquivalenz Beweisen}
	Die linken Seiten liegen bereits in DNF vor, daher müssen jeweils die rechten Seiten noch umgewandelt werden.\\\\
	\begin{minipage}[t]{0.5\textwidth}
		a)
		\begin{align*}
			&\phantom{=} x_1x_2 + \nyet{x_1}x_3 + \nyet{x_2} \nyet{x_3} \\
			&= \nyet{(\nyet{x_1}+\nyet{x_2}) \cdot (x_1 + \nyet{x_3}) \cdot (x_2 + x_3)} \\
			&\stackrel{\text{P8}}{=} \nyet{\nyet{x_1}+\nyet{x_2}} + \nyet{(x_1 + \nyet{x_3}) \cdot (x_2 + x_3)} \\
			&\stackrel{\text{P8}}{=} \nyet{\nyet{x_1}+\nyet{x_2}} + \nyet{x_1 + \nyet{x_3}} + \nyet{x_2 + x_3} \\
			&\stackrel{\text{P8'}}{=} x_1 x_2 + \nyet{x_1} x_3 + \nyet{x_2} \nyet{x_3} \quad \checkmark
		\end{align*}
	\end{minipage}
	\begin{minipage}[t]{0.5\textwidth}
		b)
		\begin{align*}
			&\phantom{=} x_1 \nyet{x_2} + \nyet{x_1} x_2 \\
			&= \nyet{\nyet{x_1 \cdot \nyet{x_2 x_2}} \cdot \nyet{\nyet{x_1 x_1} \cdot x_2}} \\
			&\stackrel{\text{P8}}{=} \nyet{\nyet{x_1 \cdot \nyet{x_2 x_2}}} + \nyet{\nyet{\nyet{x_1 x_1} \cdot x_2}} \\
			&\stackrel{\text{P7}}{=} x_1 \cdot \nyet{x_2 x_2} + \nyet{x_1 x_1} \cdot x_2 \\
			&\stackrel{\text{P3}}{=} x_1 \nyet{x_2} + \nyet{x_1} x_2  \quad \checkmark
		\end{align*}
	\end{minipage}


	\subsection*{Aufgabe 4: Minimierung macht alles einfacher!}
    \begin{itemize}
    	\item[a)]
    	\begin{align*}
    		g(x_1,x_2) & = \nyet{\nyet{x_1 \cdot x_2} \cdot x_1} \\
    		& \stackrel{\text{P8}}{=} \nyet{\nyet{x_1 \cdot x_2}} + \nyet{x_1} \\
    		& \stackrel{\text{P7}}{=} (x_1 \cdot x_2) + \nyet{x_1} \\
    		& \stackrel{\text{P4'}}{=} (x_1 + \nyet{x_1}) \cdot (\nyet{x_1} + x_2) \\
    		& \stackrel{\text{P9'}}{=} 1 \cdot (\nyet{x_1} + x_2) \\
    		& \stackrel{\text{P5}}{=} \nyet{x_1} + x_2
    	\end{align*}
    	\item[b)]
    	\begin{align*}
    		h(x_1,x_2,x_3,x_4) & = (x_1 \cdot x_2) + (x_1 \cdot x_3) + x_1 \cdot (x_2 + x_3 \cdot x_4) + x_1 \\
    		& \stackrel{\text{P4}}{=} (x_1 \cdot x_2) + (x_1 \cdot x_3) + (x_1 \cdot x_2) + (x_1 \cdot x_3 \cdot x_4) + x_1 \\
    		& \stackrel{\text{P3'}}{=} (x_1 \cdot x_2) + (x_1 \cdot x_3) + (x_1 \cdot x_3 \cdot x_4) + x_1 \\
    		& \stackrel{\text{P4}}{=} (x_1 \cdot x_2) + (x_1 \cdot x_3) \cdot (1 + x_4) + x_1 \\
    		& \stackrel{\text{P6'}}{=} (x_1 \cdot x_2) + (x_1 \cdot x_3) \cdot 1 + x_1 \\
    		& \stackrel{\text{P5}}{=} (x_1 \cdot x_2) + (x_1 \cdot x_3) + x_1 \\
    		& \stackrel{\text{P4}}{=} x_1 \cdot (x_2 + x_3 + 1) \\
    		& \stackrel{\text{P6'}}{=} x_1 \cdot 1 \\
    		& \stackrel{\text{P5}}{=} x_1
    	\end{align*}
    	\item[c)]
    	\begin{align*}
    		k(x_1,x_2,x_3) & = ((x_1 + x_3 \cdot (x_2 +x_3)) \cdot 1) \cdot 1 \\
    		& \stackrel{\text{P5}}{=} (x_1 + x_3 \cdot (x_2 +x_3)) \cdot 1 \\
    		& \stackrel{\text{P5}}{=} x_1 + x_3 \cdot (x_2 +x_3) \\
    		& \stackrel{\text{P11'}}{=} x_1 + x_3
    	\end{align*}
    \end{itemize}
    
    
    \subsection*{Aufgabe 5: Kanonen? Nein kanonisch!}
    \[
    f(x_2,x_1,x_0) = \begin{cases}
    	1 \text{ falls der Dezimalwert von }(x_2,x_1,x_0) \text{ mod 2 = 0} \\
    	0 \text{ sonst}
    	\end{cases}    
    \]\\\\
    $(x_2,x_1,x_0)$ eine vorzeichenlose Binärzahl, $x_0$ ist LSB.\\
    $(x_2,x_1,x_0)$ mod 2 = 0, wenn $(x_2,x_1,x_0)$ eine gerade Zahl ist d.h. $x_0 = 0$
    \begin{itemize}
    	\item[a)]
    	\begin{tabular}[t]{lll|l}
    		$x_2$&$x_1$&$x_0$&$f$ \\
    		\hline
    		0&0&0&1 \\
    		0&0&1&0 \\
    		0&1&0&1 \\
    		0&1&1&0 \\
    		1&0&0&1 \\
    		1&0&1&0 \\
    		1&1&0&1 \\
    		1&1&1&0 \\
    	\end{tabular}
    	\item[b)]DKNF von $f$: $$f(x_2,x_1,x_0) = \nyet{x_2}\cdot\nyet{x_1}\cdot\nyet{x_0} + \nyet{x_2}\cdot x_1\cdot\nyet{x_0} + x_2\cdot\nyet{x_1}\cdot\nyet{x_0} + x_2\cdot x_1\cdot\nyet{x_0}$$
    	KKNF von $f$: $$f(x_2,x_1,x_0) = (x_2+x_1+\nyet{x_0})\cdot(x_2+\nyet{x_1}+\nyet{x_0})\cdot(\nyet{x_2}+x_1+\nyet{x_0})\cdot(\nyet{x_2}+\nyet{x_1}+\nyet{x_0})$$
    	\item[c)]
    	\begin{align*}
    		f(x_2,x_1,x_0) & = \nyet{x_2}\cdot\nyet{x_1}\cdot\nyet{x_0} + \nyet{x_2}\cdot x_1\cdot\nyet{x_0} + x_2\cdot\nyet{x_1}\cdot\nyet{x_0} + x_2\cdot x_1\cdot\nyet{x_0} \\
    		& \stackrel{\text{P4}}{=} \nyet{x_2} \cdot\nyet{x_0}\cdot(\nyet{x_1} + x_1) + x_2 \cdot\nyet{x_0}\cdot(\nyet{x_1} + x_1) \\
    		& \stackrel{\text{P9'}}{=} \nyet{x_2} \cdot\nyet{x_0} + x_2 \cdot\nyet{x_0} \\
    		& \stackrel{\text{P4}}{=} \nyet{x_0}\cdot(\nyet{x_2} + x_2) \\
    		& \stackrel{\text{P9'}}{=} \nyet{x_0}
    	\end{align*}
	\end{itemize}
	

	\subsection*{Aufgabe 6: Nicht oder und, oder?}
	Wir machen uns folgende Tatsachen zunutze (die Nummerierung wird bei der Gleichungsumformung unten wiederverwendet):
	\begin{enumerate}
		\item $\nyet{a} = a \nand 1 = a \nor 0$
		\item $a \andd b = \nyet{a \nand b} = \nyet{a} \nor \nyet{b}$
		\item $a \orr b = \nyet{a} \nand \nyet{b} = \nyet{a \nor b}$
	\end{enumerate}
	\begin{enumerate}
		\item[a)]
		\begin{align*}
			x_1 \oplus x_2 &= (x_1 \nyet{x_2}) + (\nyet{x_1} x_2) \\\\
			&\stackrel{\text{P8}}{=} \nyet{\nyet{(x_1 \nyet{x_2})} \cdot \nyet{(\nyet{x_1} x_2)}}\\
			&\stackrel{\text{2.}}{=} \nyet{(x_1 \nand \nyet{x_2}) \cdot (\nyet{x_1} \nand x_2)} \\
			&\stackrel{\text{2.}}{=} (x_1 \nand \nyet{x_2}) \nand (\nyet{x_1} \nand x_2) \\
			&\stackrel{\text{1.}}{=} (x_1 \nand (x_2 \nand 1)) \nand ((x_1 \nand 1) \nand x_2) \\\\
			&\stackrel{\text{P7}}{=} \nyet{\nyet{(x_1 \nyet{x_2}) + (\nyet{x_1} x_2)}} \: \stackrel{\text{P8'}}{=} \: \nyet{\nyet{(x_1 \nyet{x_2})} \cdot \nyet{(\nyet{x_1} x_2)}} \\
			&\stackrel{\text{P8}}{=} \nyet{(\nyet{x_1} x_2) \cdot (x_1 \nyet{x_2})} \: \stackrel{\text{P4}}{=} \: \nyet{\nyet{x_1}x_1 + \nyet{x_1}\nyet{x_2} + ab + x_2\nyet{x_2}} \\
			&\stackrel{\text{P9}}{=} \nyet{\nyet{x_1}\nyet{x_2} + ab} \stackrel{\text{3.}}{=} (\nyet{x_1}\nyet{x_2}) \nor (ab) \\
			&\stackrel{\text{3.}}{=} (x_1 \nor x_2) \nor (\nyet{x_1} \nor \nyet{x_2}) \\
			&\stackrel{\text{1.}}{=} (x_1 \nor x_2) \nor ((x_1 \nor 0) \nor (x_2 \nor 0))
		\end{align*}
		% \begin{align*}
		% 	&\phantom{=} x_1 \oplus x_2 \\
		% 	&= (x_1 \andd \nyet{x_2}) \orr (\nyet{x_1} \andd x_2) \\
		% 	&= (\nyet{x_1} \nor x_2) \orr (x_1 \nor \nyet{x_2}) \\
		% 	&= ((x_1 \nor 0) \nor x_2) \orr (x_1 \nor (x_2 \nor 0)) \\
		% 	&= \nyet{((x_1 \nor 0) \nor x_2) \nor (x_1 \nor (x_2 \nor 0))} \\
		% 	&= (((x_1 \nor 0) \nor x_2) \nor (x_1 \nor (x_2 \nor 0))) \nor 0 \\
		% 	\\
		% 	&= x_1 \andd \nyet{x_2} \orr \nyet{x_1} \andd x_2 \\
		% 	&= \nyet{x_1 \nand \nyet{x_2}} \orr \nyet{\nyet{x_1} \nand x_2} \\
		% 	&= \nyet{x_1 \nand (x_2 \nand 1)} \orr \nyet{(x_1 \nand 1) \nand x_2} \\
		% 	&= ((x_1 \nand (x_2 \nand 1)) \nand 1) \orr (((x_1 \nand 1) \nand x_2) \nand 1) \\
		% 	&= \nyet{((x_1 \nand (x_2 \nand 1)) \nand 1)} \nand \nyet{(((x_1 \nand 1) \nand x_2) \nand 1)} \\
		% 	&= (x_1 \nand (x_2 \nand 1)) \nand ((x_1 \nand 1) \nand x_2) \\
		% \end{align*}
		\item[b)]
		\begin{align*}
			&\phantom{=} (x_1 \andd \nyet{x_2}) \orr (\nyet{x_2} \andd \nyet{x_3}) \orr (x_3 \andd \nyet{x_0}) \orr (x_0 \andd \nyet{x_1}) \\
			&\stackrel{\text{P2', 2.}}{=} \nyet{(x_1 \andd \nyet{x_2})} \nand \nyet{(\nyet{x_2} \andd \nyet{x_3})} \nand \nyet{(x_3 \andd \nyet{x_0})} \nand \nyet{(x_0 \andd \nyet{x_1})} \\
			&\stackrel{\text{2.}}{=} \nyet{\nyet{(x_1 \nand \nyet{x_2})}} \nand \nyet{\nyet{(\nyet{x_2} \nand \nyet{x_3})}} \nand \nyet{\nyet{(x_3 \nand \nyet{x_0})}} \nand \nyet{\nyet{(x_0 \nand \nyet{x_1})}} \\
			&\stackrel{\text{P7}}{=} (x_1 \nand \nyet{x_2}) \nand (\nyet{x_2} \nand \nyet{x_3}) \nand (x_3 \nand \nyet{x_0}) \nand (x_0 \nand \nyet{x_1}) \\
			&\stackrel{\text{1.}}{=} (x_1 \nand (x_2 \nand 1)) \nand ((x_2 \nand 1) \nand (x_3 \nand 1)) \nand \\ 
			& \phantom{=} (x_3 \nand (x_0 \nand 1)) \nand (x_0 \nand (x_1 \nand 1)) 
		\end{align*} 
	\end{enumerate}


	\subsection*{Aufgabe 7: KV \& Shannon}
	\begin{enumerate}
		\item[a)]KV-Diagramm:\\
		\begin{karnaugh-map}[4][2][1][$x_2, x_0$][$x_1$]
			\manualterms{0,1,0,0,1,1,1,0}
			\implicant{1}{5}
			\implicantedge{4}{4}{6}{6}
		\end{karnaugh-map}\\
		$\Rightarrow f(x_0, x_1, x_2) = \textcolor{red}{x_0 \nyet{x_2}} + \textcolor{green}{\nyet{x_0} x_1}$ \\\\
		Shannon-Zerlegung:
		\begin{align*}
			f(x_0, x_1, x_2) &= x_0 \nyet{x_2} + \nyet{x_0} x_1 \\\\
			&= \nyet{x_0} \cdot f_{\nyet{x_0}}(x) + x_0 \cdot f_{x_0}(x) \\
			&= \nyet{x_0} \cdot (0 \cdot \nyet{x_2} + 1 \cdot x_1) + x_0 \cdot (1 \cdot \nyet{x_2} + 0 \cdot x_1) \\
			&= \nyet{x_0} x_1 + x_0 \nyet{x_2} \\\\
			&= \nyet{x_1} \cdot f_{\nyet{x_1}}(x) + x_1 \cdot f_{x_1}(x) \\
			&= \nyet{x_1} \cdot (\nyet{x_0} \cdot 0 + x_0 \nyet{x_2}) + x_1 \cdot (\nyet{x_0} \cdot 1 + x_0 \nyet{x_2}) \\
			&= x_0 \nyet{x_1} \nyet{x_2} + \nyet{x_0} x_1 + x_0 x_1 \nyet{x_2} \\\\
			&= \nyet{x_2} \cdot f_{\nyet{x_2}}(x) + x_2 \cdot f_{x_2}(x) \\
			&= \nyet{x_2} \cdot (x_0 \nyet{x_1} \cdot 1 + \nyet{x_0} x_1 + x_0 x_1 \cdot 1) + x_2 \cdot (x_0 \nyet{x_1} \cdot 0 + \nyet{x_0} x_1 + x_0 x_1 \cdot 0) \\
			&= x_0 \nyet{x_1} \nyet{x_2} + \nyet{x_0} x_1 \nyet{x_2} + x_0 x_1 \nyet{x_2} + \nyet{x_0} x_1 x_2 \quad \checkmark
		\end{align*}

		\item[b)]
		KV-Diagramm:\\
		\begin{karnaugh-map}[4][4][1][$x_2, x_0$][$x_3, x_1$]
			\manualterms{0,1,1,0,1,0,1,0,0,1,1,0,1,0,1,0}
			\implicantedge{1}{1}{9}{9}
			\implicantedge{4}{12}{6}{14}
			\implicant{2}{10}
		\end{karnaugh-map}\\
		$\Rightarrow g(x_0, x_1, x_2, x_3) = \textcolor{red}{x_0 \nyet{x_1} \nyet{x_2}} + \textcolor{green}{\nyet{x_0} x_1} + \textcolor{yellow}{\nyet{x_0} x_2}$ \\\\
		Shannon-Zerlegung:
		\begin{align*}
			&\phantom{=} g(x_0, x_1, x_2, x_3) \\
			&= x_0 \nyet{x_1} \nyet{x_2} + \nyet{x_0} x_1 + \nyet{x_0} x_2 \\\\
			&= \nyet{x_0} \cdot g_{\nyet{x_0}}(x) + x_0 \cdot g_{x_0}(x) \\
			&= \nyet{x_0} \cdot (0\cdot \nyet{x_1} \nyet{x_2} + 1\cdot x_1 + 1\cdot x_2) + x_0 \cdot (1\cdot \nyet{x_1} \nyet{x_2} + 0\cdot x_1 + 0\cdot x_2) \\
			&= \nyet{x_0} x_1 + \nyet{x_0} x_2 +  x_0 \nyet{x_1} \nyet{x_2} \\\\
			&= \nyet{x_1} \cdot g_{\nyet{x_1}}(x) + x_1 \cdot g_{x_1}(x) \\
			&= \nyet{x_1} \cdot (\nyet{x_0} \cdot 0 + \nyet{x_0} x_2 +  x_0 \cdot 1 \cdot \nyet{x_2}) + x_1 \cdot (\nyet{x_0} \cdot 1 + \nyet{x_0} x_2 +  x_0 \cdot 0 \cdot \nyet{x_2}) \\
			&= \nyet{x_0} \nyet{x_1} x_2 + x_0 \nyet{x_1} \nyet{x_2} + \nyet{x_0} x_1 + \nyet{x_0} x_1 x_2 \\\\
			&= \nyet{x_2} \cdot g_{\nyet{x_2}}(x) + x_2 \cdot g_{x_2}(x) \\
			&= \nyet{x_2} \cdot (\nyet{x_0} \nyet{x_1} \cdot 0 + x_0 \nyet{x_1} \cdot 1 + \nyet{x_0} x_1 + \nyet{x_0} x_1 \cdot 0) + x_2 \cdot (\nyet{x_0} \nyet{x_1} \cdot 1 + x_0 \nyet{x_1} \cdot 0 + \nyet{x_0} x_1 + \nyet{x_0} x_1 \cdot 1) \\
			&= x_0 \nyet{x_1} \nyet{x_2} + \nyet{x_0} x_1 \nyet{x_2} + \nyet{x_0} \nyet{x_1} x_2 + \nyet{x_0} x_1 x_2 + \nyet{x_0} x_1 x_2 \\\\
			&= \nyet{x_3} \cdot g_{\nyet{x_3}}(x) + x_3 \cdot g_{x_3}(x) \\
			&= \nyet{x_3} \cdot (x_0 \nyet{x_1} \nyet{x_2} + \nyet{x_0} x_1 \nyet{x_2} + \nyet{x_0} \nyet{x_1} x_2 + \nyet{x_0} x_1 x_2 + \nyet{x_0} x_1 x_2) + x_3 \cdot (x_0 \nyet{x_1} \nyet{x_2} + \nyet{x_0} x_1 \nyet{x_2} + \nyet{x_0} \nyet{x_1} x_2 + \nyet{x_0} x_1 x_2 + \nyet{x_0} x_1 x_2) \\
			&= x_0 \nyet{x_1} \nyet{x_2} \nyet{x_3} + \nyet{x_0} x_1 \nyet{x_2} \nyet{x_3} + \nyet{x_0} \nyet{x_1} x_2 \nyet{x_3} + \nyet{x_0} x_1 x_2 \nyet{x_3} + \nyet{x_0} x_1 x_2 \nyet{x_3} \\
			&+ x_0 \nyet{x_1} \nyet{x_2} x_3 + \nyet{x_0} x_1 \nyet{x_2} x_3 + \nyet{x_0} \nyet{x_1} x_2 x_3 + \nyet{x_0} x_1 x_2 x_3 + \nyet{x_0} x_1 x_2 x_3 \\\\
			&= x_0 \nyet{x_1} \nyet{x_2} \nyet{x_3} + \nyet{x_0} \nyet{x_1} x_2 \nyet{x_3} + \nyet{x_0} x_1 \nyet{x_2} \nyet{x_3} \\
			&+ \nyet{x_0} x_1 x_2 \nyet{x_3} + \nyet{x_0} x_1 \nyet{x_2} x_3 + \nyet{x_0} x_1 x_2 x_3 \\
			&+ x_0 \nyet{x_1} \nyet{x_2} x_3 + \nyet{x_0} \nyet{x_1} x_2 x_3 \quad \checkmark
		\end{align*}
	\end{enumerate}
\end{document}