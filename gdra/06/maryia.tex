\documentclass{article}

\usepackage[left=3cm,right=2cm,top=2.5cm,bottom=2cm]{geometry}
\usepackage[utf8]{inputenc}
\usepackage[T1]{fontenc}
\usepackage[ngerman]{babel}
\usepackage{amsmath}
\usepackage{amsfonts}
\usepackage[european,noarrowmos]{circuitikz}
\usepackage{xcolor}
\usepackage{karnaugh-map}

\title{Grundlagen der Rechnerarchitektur: Übungsblatt 6}
\author{Alexander Waldenmaier, Maryia Masla}

\begin{document}
    \maketitle

    \subsection*{Aufgabe 1: Quine McCluskey}
    Wahrheitstabelle der Funktionen $f(x_2,x_1,x_0)$ und $g(x_2,x_1,x_0)$:\\
    \begin{center}
    	\begin{tabular}{ccc|cc}
    		$x_2$&$x_1$&$x_0$&$f()$&$g()$ \\
    		\hline
    		0& 0& 0&1 &0 \\
    		0& 0& 1& 0&1 \\
    		0& 1& 0& 0&0 \\
    		0& 1& 1& 1&0 \\
    		1& 0& 0& 0&1 \\
    		1& 0& 1& 1&1 \\
    		1& 1& 0& 1&0 \\
    		1& 1& 1& 1&1 \\
    	\end{tabular}
    \end{center}
    \begin{itemize}
    	\item[a)]$f_{DKNF}(x_2,x_1,x_0) = \overline{x_2x_1x_0} + \overline{x_2}x_1x_0 + x_2\overline{x_1}x_0 + x_2x_1\overline{x_0} + x_2x_1x_0$\\\\
    	\begin{minipage}{0.4\textwidth}
    		\begin{tikzpicture}
    			\node[rectangle,draw] at (0,2) {1. $\overline{x_2x_1x_0}$};
    			\node[] at (0,1.5) {2. $\overline{x_2}x_1x_0$};
    			\node[] at (0,1) {3. $x_2\overline{x_1}x_0$};
    			\node[] at (0,0.5) {4. $x_2x_1\overline{x_0}$};
    			\node[] at (0,0) {5. $x_2x_1x_0$};
    			
    			\node[rectangle,draw] at (3,1.5) {$x_1x_0$};
    			\node[rectangle,draw] at (3,1) {$x_2x_0$};
    			\node[rectangle,draw] at (3,0.5) {$x_2x_1$};
    			
    			\draw[] (0.8,0) -- (2.5,1.5) -- (0.8,1.5);
    			\draw[] (0.8,0) -- (2.5,1) -- (0.8,1);
    			\draw[] (0.8,0) -- (2.5,0.5) -- (0.8,0.5);
    		\end{tikzpicture}
    	\end{minipage}
    	\begin{minipage}{0.3\textwidth}
    		\begin{tabular}{c|c|c|c|c|c|}
    			                       & 1 & 2 & 3 & 4 & 5 \\\hline
    			$\overline{x_2x_1x_0}$ & + &   &   &   &   \\\hline
    			       $x_1x_0$        &   & + &   &   & + \\\hline
    			       $x_2x_0$        &   &   & + &   & + \\\hline
    			       $x_2x_1$        &   &   &   & + & + \\\hline
    		\end{tabular}
    	\end{minipage}\\\\
    	$\Rightarrow g_{QMC}(x_2,x_1,x_0) = \overline{x_2x_1x_0} + x_1x_0 + x_2x_0 + x_2x_1$\\
    	\item[b)]$g_{DKNF}(x_2,x_1,x_0) = \overline{x_2x_1}x_0 + x_2\overline{x_1x_0} + x_2\overline{x_1}x_0 + x_2x_1x_0$\\\\
    	\begin{minipage}{0.4\textwidth}
    		\begin{tikzpicture}
    			\node[] at (0,1.5) {1. $\overline{x_2x_1}x_0$};
    			\node[] at (0,1) {2. $x_2\overline{x_1x_0}$};
    			\node[] at (0,0.5) {3. $x_2\overline{x_1}x_0$};
    			\node[] at (0,0) {4. $x_2x_1x_0$};
    			
    			\node[rectangle, draw] at (3,1.54) {$\overline{x_1}x_0$};
    			\node[rectangle,draw] at (3,1) {$x_2\overline{x_1}$};
    			\node[rectangle,draw] at (3,0.5) {$x_2x_0$};
    			
    			\draw[] (0.8,0.5) -- (2.5,1.5) -- (0.8,1.5);
    			\draw[] (0.8,0.5) -- (2.5,1) -- (0.8,1);
    			\draw[] (0.8,0) -- (2.5,0.5) -- (0.8,0.5);
    		\end{tikzpicture}
    	\end{minipage}
    	\begin{minipage}{0.3\textwidth}
    		\begin{tabular}{c|c|c|c|c|}
    			                    & 1 & 2 & 3 & 4 \\ \hline
    			$\overline{x_1}x_0$ & + &   & + &   \\ \hline
    			$x_2\overline{x_1}$ &   & + & + &   \\ \hline
    			     $x_2x_0$       &   &   & + & + \\ \hline
    		\end{tabular}
    	\end{minipage}\\\\
    	$\Rightarrow g_{QMC}(x_2,x_1,x_0) = \overline{x_1}x_0 + x_2\overline{x_1} + x_2x_0$\newpage
    	\item[c)]Zur Minimierung der Gatterschaltung von f() $\overline{x_2}\cdot \overline{x_1}\cdot \overline{x_0}$ als $\overline{x_2+x_1+x_0}$ umschreiben\\\\
    	\begin{circuitikz}
    		\ctikzset{tripoles/european not symbol=ieee circle}
    		\draw (0,0) node[](x2){$x_2$} ++
    		(0,-1) node[](x1){$x_1$} ++
    		(0,-1) node[](x0){$x_0$};
    		
    		\draw (x2) to[short, -*] ++(1,0)
    		to[short, -] ++(0,3) -- ++(0.5,0) node[and port, anchor=in 1](AND2-1){}
    		(x1) to[short, -*] (x1 -| AND2-1.in 2) coordinate(w1) -- (AND2-1.in 2)
    		
    		to[short, -*] ++(0,-0.5) coordinate(tmp) -- (tmp -| AND2-1.out) node[and port, anchor=bin 1](AND1-0){}
    		(x0) to[short, -*] (x0 -| AND1-0.in 2) coordinate(w0) -- (AND1-0.in 2)
    		
    		to[short, -*] ++(0,-0.5) coordinate(tmp) -- (tmp -| AND1-0.out) node[and port, anchor=bin 1](AND2-0){}
    		(x2) to[short, -*] (x2 -| AND2-0.in 2) coordinate(w2) -- (AND2-0.in 2)
    		
    		(AND1-0.out) -- (AND1-0.out -| AND2-0.bout) node[or port, anchor=bin 2](OR1){}
    		(AND2-1.out) -| (OR1.in 1)
    		
    		(AND2-0.out) -- (AND2-0.out -| OR1.out) node[or port, anchor=in 2](OR2){}
    		(OR1.out) -| (OR2.in 1)
    		
    		
    		
    		(w1) -- (w1 -| AND2-0.in 2) node[or port, anchor=in 2](OR3){}
    		(w2) -| (OR3.in 1)
    		
    		(OR3.out) -- (OR3 -| OR2.in 1) node[or port, anchor=in 1](OR4){}
    		(w0) -| (OR4.in 2)
    		
    		(OR4.out) to[inline not] ++(2,0) coordinate(not) ++(2,0) coordinate(tmp)
    		(w2 -| tmp) node[or port](OR5){}
    		(not) -| (OR5.in 2)
    		(OR2.out) -| (OR5.in 1)
    		(OR5.out) ++(1,0)node[](f){$f(x_2,x_1,x_0)$}
    		;
    	\end{circuitikz}\\\\
    	Gatterschaltung von g():\\\\
    	\begin{circuitikz}
    		\ctikzset{tripoles/european not symbol=ieee circle}
    		\draw (0,0) node[](x2){$x_2$} ++
    		(0,-1) node[](x1){$x_1$} ++
    		(0,-1) node[](x0){$x_0$};
    		
    		\draw (x1) to[inline not] ++(2.5,0) -- ++(0,2.5)
    		-- ++(0,0) node[and port, anchor=in 2](AND2-1){}
    		
    		(x2) to[short, -*] ++(2,0) |- (AND2-1.in 1)
    		
    		(AND2-1.in 2) to[short, -*] ++(0,-0.5) coordinate(tmp) -- (tmp -| AND2-1.out) node[and port, anchor=bin 1](AND1-0){}
    		
    		(x0) to[short, -*] (x0 -| AND1-0.in 2) -- (AND1-0.in 2)
    		
    		(AND1-0.out) -- ++(0,0) node[or port, anchor=in 2](OR1){}
    		(AND2-1.out) -| (OR1.in 1)
    		
    		(x1 -| OR1) node[and port](AND2-0){}
    		(x2) -| (AND2-0.in 1)
    		(x0) -| (AND2-0.in 2)
    		
    		(x2 -| OR1) ++(2.5,0) node[or port](OR2){}
    		(OR1.out) -| (OR2.in 1)
    		(AND2-0.out) -| (OR2.in 2)
    		(OR2.out) ++(1,0)node[](g){$g(x_2,x_1,x_0)$}
    		;
    	\end{circuitikz}
    \end{itemize}
    
    
    \subsection*{Aufgabe 2: Aiken-Code}
    \begin{itemize}
    	\item[a)]X-Don't care\\\\
    	\begin{tabular}[t]{c|cccc|cccc}
    		Dezimal &   \multicolumn{4}{c}{Binär}   & \multicolumn{4}{c}{Aiken-Code} \\
    		d       & $x_1$ & $x_2$ & $x_3$ & $x_4$ & $y_1$ & $y_2$ & $y_3$ & $y_4$  \\ \hline
    		0       & 0     & 0     & 0     & 0     & 0     & 0     & 0     & 0      \\
    		1       & 0     & 0     & 0     & 1     & 0     & 0     & 0     & 1      \\
    		2       & 0     & 0     & 1     & 0     & 0     & 0     & 1     & 0      \\
    		3       & 0     & 0     & 1     & 1     & 0     & 0     & 1     & 1      \\
    		4       & 0     & 1     & 0     & 0     & 0     & 1     & 0     & 0      \\
    		5       & 0     & 1     & 0     & 1     & 1     & 0     & 1     & 1      \\
    		6       & 0     & 1     & 1     & 0     & 1     & 1     & 0     & 0      \\
    		7       & 0     & 1     & 1     & 1     & 1     & 1     & 0     & 1      \\
    		8       & 1     & 0     & 0     & 0     & 1     & 1     & 1     & 0      \\
    		9       & 1     & 0     & 0     & 1     & 1     & 1     & 1     & 1      \\
    		10      & 1     & 0     & 1     & 0     & X     & X     & X     & X      \\
    		11      & 1     & 0     & 1     & 1     & X     & X     & X     & X      \\
    		12      & 1     & 1     & 0     & 0     & X     & X     & X     & X      \\
    		13      & 1     & 1     & 0     & 1     & X     & X     & X     & X      \\
    		14      & 1     & 1     & 1     & 0     & X     & X     & X     & X      \\
    		15      & 1     & 1     & 1     & 1     & X     & X     & X     & X
    	\end{tabular}\\
    	\item[b)]$y_1(x_1,x_2,x_3,x_4)=\overline{x_1}x_2\overline{x_3}x_4 + \overline{x_1}x_2x_3\overline{x_4} + \overline{x_1}x_2x_3x_4 + x_1\overline{x_2x_3x_4} + x_1\overline{x_2x_3}x_4$\\
    	$y_2(x_1,x_2,x_3,x_4)=\overline{x_1}x_2\overline{x_3x_4} + \overline{x_1}x_2x_3\overline{x_4} + \overline{x_1}x_2x_3x_4 + x_1\overline{x_2x_3x_4} + x_1\overline{x_2x_3}x_4$\\
    	$y_3(x_1,x_2,x_3,x_4)=\overline{x_1x_2}x_3\overline{x_4} + \overline{x_1x_2}x_3x_4 + \overline{x_1}x_2\overline{x_3}x_4 + x_1\overline{x_2x_3x_4} + x_1\overline{x_2x_3}x_4$\\
    	$y_4(x_1,x_2,x_3,x_4)=\overline{x_1x_2x_3}x_4 + \overline{x_1x_2}x_3x_4 + \overline{x_1}x_2\overline{x_3}x_4 + \overline{x_1}x_2x_3x_4 + x_1\overline{x_2x_3}x_4$\\
    	\item[c)]Minimieren mittels KV-Diagramm:\\\\
    	\begin{minipage}[t]{0.45\textwidth}
    		$y_1(x_1,x_2,x_3,x_4)=x_1 + x_2x_3 + x_2x_4$\\\\
    		\begin{karnaugh-map}[4][4][1][$x_3x_4$][$x_1x_2$]
    			\minterms{5,6,7,8,9}
    			\terms{10,11,12,13,14,15}{X}
    			\autoterms[0]
    			\implicant{12}{10}
    			\implicant{5}{15}
    			\implicant{7}{14}
    		\end{karnaugh-map}
    	\end{minipage}
    	\begin{minipage}[t]{0.45\textwidth}
    		$y_2(x_1,x_2,x_3,x_4)=x_1 + x_2x_3 + x_2\overline{x_4}$\\\\
    		\begin{karnaugh-map}[4][4][1][$x_3x_4$][$x_1x_2$]
    			\minterms{4,6,7,8,9}
    			\terms{10,11,12,13,14,15}{X}
    			\autoterms[0]
    			\implicant{12}{10}
    			\implicantedge{4}{12}{6}{14}
    			\implicant{7}{14}
    		\end{karnaugh-map}
    	\end{minipage}\\
    	\begin{minipage}[t]{0.45\textwidth}
    		$y_3(x_1,x_2,x_3,x_4)=x_1 + \overline{x_2}x_3 + x_2\overline{x_3}x_4$\\\\
    		\begin{karnaugh-map}[4][4][1][$x_3x_4$][$x_1x_2$]
    			\minterms{2,3,5,8,9}
    			\terms{10,11,12,13,14,15}{X}
    			\autoterms[0]
    			\implicant{12}{10}
    			\implicantedge{3}{2}{11}{10}
    			\implicant{5}{13}
    		\end{karnaugh-map}
    	\end{minipage}
    	\begin{minipage}[t]{0.45\textwidth}
    		$y_4(x_1,x_2,x_3,x_4)=x_4$\\\\
    		\begin{karnaugh-map}[4][4][1][$x_3x_4$][$x_1x_2$]
    			\minterms{1,3,5,7,9}
    			\terms{10,11,12,13,14,15}{X}
    			\autoterms[0]
    			\implicant{1}{11}
    		\end{karnaugh-map}
    	\end{minipage}
    	\item[d)]
    	\begin{align*}
    		y_1(x_1,x_2,x_3,x_4) & = x_1 \oplus x_2x_3 \oplus x_2x_4                              \\
    		                     &                                                                \\
    		y_2(x_1,x_2,x_3,x_4) & = x_1 \oplus x_2x_3 \oplus x_2\overline{x_4}                   \\
    		                     & = x_1 \oplus x_2x_3 \oplus x_2(1 \oplus x_4)                   \\
    		                     & = x_1 \oplus x_2 \oplus x_2x_3 \oplus x_2x_4                   \\
    		                     &                                                                \\
    		y_3(x_1,x_2,x_3,x_4) & = x_1 + \overline{x_2}x_3 + x_2\overline{x_3}x_4               \\
    		                     & = x_1 \oplus \overline{x_2}x_3 \oplus x_2\overline{x_3}x_4    \\
    		                     & = x_1 \oplus x_3(1 \oplus x_2) \oplus x_2x_4(1 \oplus x_3)    \\
    		                     & = x_1 \oplus x_3 \oplus x_2x_3 \oplus x_2x_4 \oplus x_2x_3x_4 \\
    		                     &                                                                \\
    		y_4(x_1,x_2,x_3,x_4) & = x_4
    	\end{align*}
    	\begin{circuitikz}
    		\ctikzset{tripoles/european not symbol=ieee circle}
    		\draw (0,0) node[](x1){$x_1$} ++
    		(0,-2) node[](x2){$x_2$} ++
    		(0,-2) node[](x3){$x_3$} ++
    		(0,-2) node[](x4){$x_4$}
    		
    		(x3) -- ++(2,0) to[short, *-] ++(0,0.75) node[and port, anchor=in 2](AND23){}
    		(x2) to[short, -*] (x2 -| AND23.in 1) -- (AND23.in 1)
    		
    		(x4) -- ++(1,0) to[short, *-] ++(0,0.75) coordinate(tmp) -- (tmp -| AND23.in 1) node[and port, anchor=in 2](AND24){}
    		(x2) to[short, -*] ++(1,0) |- (AND24.in 1)
    		
    		(AND23.out) to[short, -*] ++(0.5,0) node[and port, anchor=in 1](AND234){}
    		(x4) to[short, -*] (x4 -| AND234.in 2) -- (AND234.in 2)
    		
    		(x1) -- (x1 -| AND234.in 1) node[xor port, anchor=in 1](XOR1){}
    		(AND23) -| (XOR1.in 2)
    		(XOR1.out) -- ++(0.5,0) node[xor port, anchor=in 1](XOR2){}
    		(AND24.out) -| (XOR2.in 2)
    		(XOR2.out) to[short, -*] ++(2,0) coordinate(tmp) -- ++(3,0) coordinate(out) node[right](){$y_1$}
    		
    		(x2) -- (x2 -| tmp) node[xor port, anchor=in 2](XOR3){}
    		(XOR3.out) -- (XOR3 -| out) node[right](){$y_2$}
    		
    		(AND234.out) -- (AND234 -| XOR2.bout) node[xor port, anchor=bin 1](XOR4){}
    		(x3) -| (XOR4.in 2)
    		(XOR4.out) -- (XOR4 -| tmp) node[xor port, anchor=in 2](XOR5){}
    		(XOR5.out) -- (XOR5 -| out) node[right](){$y_3$}
    		
    		(tmp) to[short, -*] (XOR3.in 1 -| tmp) -- (XOR5.in 1 -| tmp)
    		
    		(x4) -- (x4 -| out) node[right](){$y_4$}
    		;
    	\end{circuitikz}
    \end{itemize}
	
	
    \subsection*{Aufgabe 3: Das RS-Flipflop}
    
    
    \subsection*{Aufgabe 4: D-FF und D-Latch}
    
    
    \end{document}