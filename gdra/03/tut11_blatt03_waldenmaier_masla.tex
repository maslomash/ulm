\documentclass{article}
\usepackage[utf8]{inputenc}
\usepackage[T1]{fontenc}
\usepackage[ngerman]{babel}
\usepackage[shortlabels]{enumitem}
\usepackage{amsmath}
\usepackage{amsfonts}
\usepackage[left=3cm,right=2cm,top=2.5cm,bottom=2cm]{geometry}
\usepackage{xcolor}
\usepackage{mathtools}

\title{Grundlagen der Rechnerarchitektur: Übungsblatt 3}
\author{Alexander Waldenmaier, Maryia Masla}

\begin{document}
    \maketitle

    \subsection*{Aufgabe 1: Multiplikation und Division}
    \begin{itemize}
        \item[a)] Anwendung der Definition 2.1.14 aus dem Skript:\\\\
        $10111011_2\cdot 1001101_2=(1_2\cdot 1000000_2+0_2\cdot 100000_2+0_2\cdot 10000_2+1_2\cdot 1000_2+1_2\cdot 100_2+0_2\cdot 10_2+1_2\cdot 1_2)\cdot 10111011_2=10111011_2\cdot 1000000_2+10111011_2\cdot 1000_2+10111011_2\cdot 100_2+10111011_2$
        \begin{center}
        	\begin{tabular}{crl}
        		  & 10111011$\cdot$1001101 & = \\ \hline
        		  &         10111011000000 &   \\
        		+ &            10111011000 &   \\
        		+ &             1011101100 &   \\
        		+ &               10111011 &   \\ \hline
        		  &         11100000111111 &
        	\end{tabular}
        \end{center}
        \item[b)] Def. 2.1.14:\\\\
        $10011010_2\cdot 111001_2=(1_2\cdot 100000_2+1_2\cdot 10000_2+1_2\cdot 1000_2+0_2\cdot 100_2+0_2\cdot 10_2+1_2\cdot 1_2)\cdot 10011010_2=10011010_2\cdot 100000_2+10011010_2\cdot 10000_2+10011010_2\cdot 1000_2+10011010_2$
        \begin{center}
        	\begin{tabular}{crl}
        	  & 10011010$\cdot$111001 & = \\ \hline
        	  &         1001101000000 &   \\
        	+ &          100110100000 &   \\
        	+ &           10011010000 &   \\
        	+ &              10011010 &   \\ \hline
        	  &        10001001001010 &
        \end{tabular}
        \end{center}
        \item[c)] Anwendung der Definition (2.1.15) aus dem Skript:\\
        \[\setlength{\arraycolsep}{0pt}
        \begin{array}{*{9}{l}}
        	1 & 0 & 0 & 1 & 1 & 0 & 1 & 0 & \ :10010=1000   \\
        	1 & 0 & 0 & 1 & 0 &   &   &   &                 \\ \cline{1-5}
        	  &   &   &   & 1 & 0 &   &   &                 \\
        	  & 0 & 0 & 0 & 0 & 0 &   &   &                 \\ \cline{2-6}
        	  &   &   &   & 1 & 0 & 1 &   &                 \\
        	  &   & 0 & 0 & 0 & 0 & 0 &   &                 \\ \cline{3-7}
        	  &   &   &   & 1 & 0 & 1 & 0 &                 \\
        	  &   &   & 0 & 0 & 0 & 0 & 0 &                 \\ \cline{4-8}
        	  &   &   &   & 1 & 0 & 1 & 0 & \ Rest        
        \end{array}
        \]
        \newpage
        \item[d)] (2.1.15):\\\\
        \[\setlength{\arraycolsep}{0pt}
       \begin{array}{*{14}{l}}
       	1 & 1 & 1 & 0 & 1 & 1 & 0 & 0 & 0 & 0 & 1 & \ :10110=1010101 \\
       	1 & 0 & 1 & 1 & 0 &   &   &   &   &   &   &                  \\ \cline{1-5}
       	  & 0 & 1 & 1 & 1 & 1 &   &   &   &   &   &                  \\
       	  & 0 & 0 & 0 & 0 & 0 &   &   &   &   &   &                  \\ \cline{2-6}
       	  &   & 1 & 1 & 1 & 1 & 0 &   &   &   &   &                  \\
       	  &   & 1 & 0 & 1 & 1 & 0 &   &   &   &   &                  \\ \cline{3-7}
       	  &   &   & 1 & 0 & 0 & 0 & 0 &   &   &   &                  \\
       	  &   &   & 0 & 0 & 0 & 0 & 0 &   &   &   &                  \\ \cline{4-8}
       	  &   &   & 1 & 0 & 0 & 0 & 0 & 0 &   &   &                  \\
       	  &   &   &   & 1 & 0 & 1 & 1 & 0 &   &   &                  \\ \cline{5-9}
       	  &   &   &   &   & 1 & 0 & 1 & 0 & 0 &   &                  \\
       	  &   &   &   &   & 0 & 0 & 0 & 0 & 0 &   &                  \\ \cline{6-10}
       	  &   &   &   &   & 1 & 0 & 1 & 0 & 0 & 1 &                  \\
       	  &   &   &   &   &   & 1 & 0 & 1 & 1 & 0 &                  \\ \cline{7-11}
       	  &   &   &   &   &   & 1 & 0 & 0 & 1 & 1 & \ Rest
       \end{array}
       \]
    \end{itemize}


    \subsection*{Aufgabe 2: Multiplizieren \& Dividieren aber schnell}
    \begin{enumerate}
        \item[a)]$01\:0010\:1010_2\cdot 00\:0000\:0010_2=10\:0101\:0100_2$
        \item[b)]$000\:0101_2\cdot 00\:0100_2=01\:0100_2$
        \item[c)]$0011\:1010\:1001_2\div 0010\:0000\:0000_2=0000\:0000\:0001_2$
        \item[d)]$0101\:0111_2\div 0000\:1000_2=0000\:1010,1110\:0000_2 = \\2^3+2^1+2^{-1}+2^{-2}+2^{-2} = 8+2+0,5+0,25+0,125=10,875_{10}$
    \end{enumerate}


    \subsection*{Aufgabe 3: Binär und doch Dezimal}
    \begin{enumerate}
        \item[a)]$377_{10}$ in BCD:\\\\
        \begin{tabular}{l|ccc} 
            $Z_{10}$ & 3    & 7    & 7 \\
            \hline
            BCD  & 0011 & 0111 & 0111
        \end{tabular}\\\\
        $\Rightarrow 377_{10} = 0011\: 0111\: 0111$ in BCD
        \item[b)]$17_{10}+13_{10}$ in BCD:\\
        \begin{minipage}[t]{0.3\textwidth}
        	\hfill \\
        	\begin{tabular}{l|cc}
        		$Z_{10}$ &  1   &  7   \\ \hline
        		BCD      & 0001 & 0111
        	\end{tabular}
        	\begin{tabular}{l|cc}
        		$Z_{10}$ &  1   &  3   \\ \hline
        		BCD      & 0001 & 0011
        	\end{tabular}
        \end{minipage}
    	\begin{minipage}[t]{0.2\textwidth}
    		\hfill \\
    		\begin{tabular}{crr}
    			  &             0001 & 0011 \\
    			+ &             0001 & 0111 \\
    			  & \color{gray}0001 &      \\ \hline
    			  &             0011 & 0000
    		\end{tabular}
    	\end{minipage}\\\\
        $\Rightarrow 17_{10} +13_{10}=0011\:0000$ in BCD
        \item[c)]$110_{10}+99_{10}$ in BCD:\\
        \begin{minipage}[t]{0.3\textwidth}
        	\hfill \\
        	\begin{tabular}{l|ccc}
        		$Z_{10}$ &  1   &  1   &  0   \\ \hline
        		BCD      & 0001 & 0001 & 0000
        	\end{tabular}
        	\begin{tabular}{l|cc}
        		$Z_{10}$ &  9   &  9   \\ \hline
        		BCD      & 1001 & 1001
        	\end{tabular}
        \end{minipage}
        \begin{minipage}[t]{0.2\textwidth}
        	\hfill \\
        	\begin{tabular}{crrr}
        		  &             0001 & 0001 & 0000 \\
        		+ &                  & 1001 & 1001 \\
        		  & \color{gray}0001 &      &      \\ \hline
        		  &             0010 & 0000 & 1001
        	\end{tabular}
        \end{minipage}\\\\
        $\Rightarrow 110_{10} +99_{10}=0010\: 0000 \:1001$ in BCD
        \newline
        \item[d)]$3_{10}\cdot 4_{10}$ in BCD:\\
        \begin{minipage}[t]{0.2\textwidth}
        	\hfill \\
        	\begin{tabular}{l|c}
        		$Z_{10}$ &  3   \\ \hline
        		BCD      & 0011
        	\end{tabular}\\
        	\begin{tabular}{l|c}
        		$Z_{10}$ &  4   \\ \hline
        		BCD      & 0100
        	\end{tabular}
        \end{minipage}
        \begin{minipage}[t]{0.4\textwidth}
        	\hfill \\
        	\begin{tabular}{crl}
        		  & $0011\cdot 0100$ & = \\ \hline
        		  &  $0000\:0010$ &   \\
        		  &  $\color{gray} 0001\:0000$ &   \\ \hline
        		  &  $0001\:0010$ &$=12_{10}$
        	\end{tabular}
        \end{minipage}
    	\begin{minipage}[t]{0.2\textwidth}
    		\hfill \\
    		\begin{tabular}{l|cc}
    			$Z_{10}$ &  1   &  2   \\ \hline
    			BCD      & 0001 & 0010
    		\end{tabular}
    	\end{minipage}\\\\
        $\Rightarrow 3\cdot 4=0011_{BCD}\cdot 0100_{BCD}=0001\:0010_{BCD}$
    \end{enumerate}


    \subsection*{Aufgabe 4: Kettenbruch}
    \begin{enumerate}
        \item[a)]$\dfrac{43}{30}=1+\dfrac{13}{30}=1+\dfrac{1}{\dfrac{30}{13}}=1+\dfrac{1}{2+\dfrac{4}{13}}=1+\dfrac{1}{2+\dfrac{1}{\dfrac{13}{4}}}=1+\dfrac{1}{2+\dfrac{1}{3+\dfrac{1}{4}}}$
        \item[b)]$\dfrac{55}{19}=2+\dfrac{17}{19}=2+\dfrac{1}{\dfrac{19}{17}}=2+\dfrac{1}{1+\dfrac{2}{17}}=2+\dfrac{1}{1+\dfrac{1}{\dfrac{17}{2}}}=2+\dfrac{1}{1+\dfrac{1}{8+\dfrac{1}{2}}}$
    \end{enumerate}


	\subsection*{Aufgabe 5: Festkommazahlen}
	\begin{enumerate}
		\item[a)]
		\begin{tabular}[t]{llllll}
			$i$ & $x_i$ & $b$     &       &       & $a_i$ \\
			\hline 
			0   & 0,453125 & $*2 = $ & 0,90625   & $\ge1$: & 0 \\
			1   & 0,90625 & $*2 = $ & 1,8125   & $\ge1$: & 1 \\
			2   & 0,8125  & $*2 = $ & 1,625   & $\ge1$: & 1 \\
			3   & 0,625  & $*2 = $ & 1,25   & $\ge1$: & 1 \\
			4   & 0,25  & $*2 = $ & 0,5   & $\ge1$: & 0 \\
			4   & 0,5  & $*2 = $ & 1,0   & $\ge1$: & 1 \\
		\end{tabular}\\\\
		$\Rightarrow 1,453125_{10} = 0000\:01,01\:1010_2$\\\\
		Es ist kein Abschneiden notwendig! 
		\item[b)] $\frac{1}{3}_{10} = 1_{10} \div 3_{10} = 0,\bar{3}$\\\\
		\begin{tabular}[t]{llllll}
			$i$ & $x_i$ & $b$     &       &       & $a_i$ \\
			\hline 
			0   & $0,\bar{3}$ & $*2 = $ & $0,\bar{6}$   & $\ge1$: & 0 \\
			1   & $0,\bar{6}$ & $*2 = $ & $1,\bar{3}$   & $\ge1$: & 1 \\
			2   & $0,\bar{3}$ & $*2 = $ & $0,\bar{6}$   & $\ge1$: & 0 \\
			\vdots   & \vdots & \vdots & \vdots   & \vdots & \vdots \\
			7  & $0,\bar{6}$ & $*2 = $ & $1,\bar{3}$   & $\ge1$: & 1 \\
			\vdots   & \vdots & \vdots & \vdots   & \vdots & \vdots 
		\end{tabular}\\\\
		$\Rightarrow \frac{1}{3}_{10} = 0000\:00,01\:0101_2$\\\\
		Um nicht abschneiden zu müssen, könnte man die Nachkommastellen als Kettenbruch darstellen. Dann müsste die Zahl 3 als einziges Kettenbruch-Glied gespeichert werden. 
	\end{enumerate}

	\subsection*{Aufgabe 6: Grey Code}
	\begin{enumerate}
		\item[a)] Farbkodierung: \textcolor{gray}{aufgefüllte Nullen}, \textcolor{red}{aufgefüllte Einsen}:
		\[\setlength{\arraycolsep}{0pt}
		\begin{array}{*{4}{c}}
			\textcolor{gray}{0} & \textcolor{gray}{0} & \textcolor{gray}{0} & 0 \\ 
			\textcolor{gray}{0} & \textcolor{gray}{0} & \textcolor{gray}{0} & 1 \\ \cline{4-4}
			\textcolor{gray}{0} & \textcolor{gray}{0} & \textcolor{red}{1} & 1 \\ 
			\textcolor{gray}{0} & \textcolor{gray}{0} & \textcolor{red}{1} & 0 \\ \cline{3-4}
			\textcolor{gray}{0} & \textcolor{red}{1} & 1 & 0 \\
			\textcolor{gray}{0} & \textcolor{red}{1} & 1 & 1 \\ 
			\textcolor{gray}{0} & \textcolor{red}{1} & 0 & 1 \\
			\textcolor{gray}{0} & \textcolor{red}{1} & 0 & 0 \\ \cline{2-4}
			\textcolor{red}{1} & 1 & 0 & 0 \\
			\textcolor{red}{1} & 1 & 0 & 1 \\
			\textcolor{red}{1} & 1 & 1 & 1 \\
			\textcolor{red}{1} & 1 & 1 & 0 \\
			\textcolor{red}{1} & 0 & 1 & 0 \\
			\textcolor{red}{1} & 0 & 1 & 1 \\
			\textcolor{red}{1} & 0 & 0 & 1 \\
			\textcolor{red}{1} & 0 & 0 & 0 
		\end{array}
		\] 
		\item[b)]
		\begin{itemize}
			\item Bei jedem Übergang von einer Zeile zur nächsten ändert jeweils nur eine Stelle ihren Wert. Die Auswirkungen von Ablesefehlern bleiben somit minimal.
			\item Durch das oben gezeigte Spiegel-Verfahren ist der Grey Code sehr einfach zu konstruieren. 
		\end{itemize}
	\end{enumerate}


	\subsection*{Aufgabe 7: Minimierung macht alles einfacher!}
	\begin{enumerate}
		\item[a)]
		\begin{align*}
			f(x_1) &= (x_1 + 0) \cdot (0 + 1) \cdot 1 \\
			&\stackrel{\text{P2}}{=} (x_1 + 0) \cdot 1 \\
			&\stackrel{\text{P5'}}{=} x_1 \cdot 1 \\
			&\stackrel{\text{P5}}{=} x_1
		\end{align*}
		\item[b)]
		\begin{align*}
			i(x_1, x_2) &= \overline{((\overline{x_1 x_2})(\overline{x_1 x_2}))} \\
			&\stackrel{\text{P3}}{=} \overline{\overline{x_1 x_2}} \\
			&\stackrel{\text{P7}}{=} x_1 x_2
		\end{align*}
		\item[c)]
		\begin{align*}
			j(x_1, x_2, x_3, x_4) &= x_1 + x_2 x_4 + x_3 + x_1 x_4 \\
			&\stackrel{\mathclap{\text{P1'}}}{=} x_1 + x_1 x_4 + x_2 x_4 + x_3 \\
			&\stackrel{\mathclap{\text{P5}}}{=} x_1 1 + x_1 x_4 + x_2 x_4 + x_3 \\
			&\stackrel{\mathclap{\text{P4}}}{=} x_1(1 + x_4) + x_2 x_4 + x_3 \\
			&\stackrel{\mathclap{\text{P6'}}}{=} x_1 x_4 + x_2 x_4 + x_3 \\
			% &\stackrel{\mathclap{\text{P4}}}{=} x_4(x_1 + x_2) + x_3
		\end{align*}
	\end{enumerate}
\end{document}