\documentclass{article}
\usepackage{amsmath}
\usepackage[utf8]{inputenc}
\usepackage[T1]{fontenc}
\usepackage[ngerman]{babel}
\usepackage[shortlabels]{enumitem}
\usepackage{amsfonts}

\title{Algorithmen und Datenstrukturen: Übung 1}
\author{Alexander Waldenmaier}

\begin{document}
    \maketitle

    \subsection*{Aufgabe 1.1}
    Zur Lösung der Gleichungen verwende ich einen Taschenrechner. 
    \begin{enumerate}
        \item[a)] 
        \begin{align*}
            50000000n &\stackrel{!}{\le} 1 \cdot 10^9 \Rightarrow n_{max} = 20 \\
            50000000n &\stackrel{!}{\le} 60 \cdot 10^9 \Rightarrow n_{max} = 1200
        \end{align*}
        \item[b)] 
        \begin{align*}
            10^7n \log_2n &\stackrel{!}{\le} 1 \cdot 10^9 \Rightarrow n_{max} = 22 \\
            10^7n \log_2n &\stackrel{!}{\le} 60 \cdot 10^9 \Rightarrow n_{max} = 643
        \end{align*} 
        \item[c)] 
        \begin{align*}
            10^6n^2 &\stackrel{!}{\le} 1 \cdot 10^9 \Rightarrow n_{max} = 31 \\
            10^6n^2 &\stackrel{!}{\le} 60 \cdot 10^9 \Rightarrow n_{max} = 244
        \end{align*} 
        \item[d)] 
        \begin{align*}
            1.5^n &\stackrel{!}{\le} 1 \cdot 10^9 \Rightarrow n_{max} = 51 \\
            1.5^n &\stackrel{!}{\le} 60 \cdot 10^9 \Rightarrow n_{max} = 61
        \end{align*}
        \item[e)] 
        \begin{align*}
            2^n &\stackrel{!}{\le} 1 \cdot 10^9 \Rightarrow n_{max} = 29 \\
            2^n &\stackrel{!}{\le} 60 \cdot 10^9 \Rightarrow n_{max} = 35
        \end{align*}
        \item[f)] 
        \begin{align*}
            n! &\stackrel{!}{\le} 1 \cdot 10^9 \Rightarrow n_{max} = 12 \\
            n! &\stackrel{!}{\le} 60 \cdot 10^9 \Rightarrow n_{max} = 13
        \end{align*}
    \end{enumerate}


    \subsection*{Aufgabe 1.2}
    \begin{enumerate}
        \item[a)]
        \begin{align*}
            \log\left(n^n\right) = n \log n = \mathcal{O}\left(n \log n\right)
        \end{align*} Die Aussage stimmt. 
        \item[b)] 
        \begin{align*}
            \log_a n = \frac{\log_b n}{\log_b a} = c \cdot \log_b n = \Theta\left(\log_b n\right), c \in \mathbb{R}
        \end{align*} Die Aussage stimmt. 
        \item[c)]
        Zu beweisen für beliebige Basen $c$: 
        \begin{align*}
            a^{\log_c b} &= b^{\log_c a}\\
            \Leftrightarrow a^{\frac{\log_b b}{\log_b c}} &= b^{\frac{\log_b a}{\log_b c}}\\
            \Leftrightarrow a^{\log_b b} &= b^{\log_b a}\\
            \Leftrightarrow a &= a, \textbf{q.e.d.}
        \end{align*} 
        Demnach kann folgende Umformung durchgeführt werden:
        \begin{align*}
            2^{\log \left(n^2\right)} = n^{2 ^{\log 2}}
        \end{align*}
        Da nicht bekannt ist, zu welcher Basis der Logarithmus definiert ist, kann keine Aussage gemacht werden. 
        \item[d)]
        Einsetzen der Bedingung liefert: 
        \begin{align*}
            \log(f(n)) = \log(\mathcal{O}(g(n))) &\stackrel{!}{=} \mathcal{O}(\log(g(n)))\\
            \log(c\cdot g(n)) &\stackrel{!}{=} \mathcal{O}(\log(g(n))), c>0\\
            \log(g(n)) + \log(c) &= \mathcal{O}(\log(g(n)))
        \end{align*} 
        Die Aussage stimmt.
        \item[e)]
        Zu widerlegen:
        \begin{align*}
            \exists c>0 \: \exists n_0>0 \: \forall n \ge n_0: 2^n \le c \cdot \sqrt{2}^n
        \end{align*}
        Widerspruchbeweis:
        \begin{align*}
            2^n &\stackrel{!}{\le} c\cdot {\sqrt 2}^n\\
            \left(\frac{2}{\sqrt{2}}\right)^n &\stackrel{!}{\le} c\\
            n \log\left(\frac{2}{\sqrt{2}}\right) &\stackrel{!}{\le} \log c\\
            n &\le \frac{\log c}{\log 2/\sqrt{2}} = n_1 = \mathrm{const.}
        \end{align*}
        Die Aussage wurde widerlegt, da $2^n$ für alle $n>n_1$ größer $\sqrt{2}^n$ ist. 
        \item[f)]
        Unter Verwendung der Regel, die in 1.2c) hergeleitet wurde gilt:  
        \begin{align*}
            f(n) &= \log(n)^{\log(n)}\\
            &= n^{\log(\log(n))} \stackrel{!}{\le} c \cdot n^k\\
            \Rightarrow n^{\log(\log(n))-k} &\stackrel{!}{\le} c\\
            \left(\log(\log(n))-k\right) \log n &\stackrel{!}{\le} \log c\\
            \log(\log(n))-k &\stackrel{!}{\le} \frac{\log c}{\log n} \\
            \log(\log(n))-\frac{\log c}{\log n} &\stackrel{!}{\le} k \\
            \lim_{n\rightarrow \infty} \log(\log(n))-\frac{\log c}{\log n} &\stackrel{!}{\le} k \\
            \infty &\stackrel{!}{\le} k
        \end{align*} 
        Die Aussage wurde widerlegt. Es existiert kein $k$ für das gilt: $f(n) = \mathcal{O}\left(n^k\right)$.
    \end{enumerate}


    \subsection*{Aufgabe 1.3}
    \begin{enumerate}
        \item[a)]
        \begin{align*}
            T_1(n) &= 5 T_1\left(\frac{n}{3}\right) + T_1\left(\frac{2n}{3}\right) +3n, m=2, k=1 \\
            &\sum_{i=1}^{m} \alpha_i^k = \frac{5}{3} + \frac{2}{3} = \frac{7}{3} > 1 \Rightarrow \textrm{Fall 3!}\\
            &\sum_{i=1}^{m} \alpha_i^c = 1 \\
            &\Leftrightarrow 5\left(\frac{1}{3}\right)^c + \left(\frac{2}{3}\right)^c = 1\\
            &\Rightarrow c = 2 \\
            &\Rightarrow T_1(n) = \Theta(n^2)
        \end{align*}
        \item[b)]
        \begin{align*}
            T_1(n) &= T_2\left(\frac{n}{4}\right) + 2T_2\left(\frac{n}{16}\right) + \sqrt{n} \\
            &= T_2\left(\frac{n}{4}\right) + 2T_2\left(\frac{n}{16}\right) + n^{1/2}, m=2, k=\frac{1}{2}\\
            &\sum_{i=1}^{m} \alpha_i^k = \sqrt{\frac{1}{4}} + 2 \sqrt{\frac{1}{16}} = \frac{1}{2} + \frac{1}{2} = 1 \Rightarrow \textrm{Fall 2!}\\
            &\Rightarrow T_2(n) = \Theta(\sqrt{n}\log n)
        \end{align*}
        \item[c)]
        \begin{align*}
            T_3(n) &= T_3\left(\frac{3n}{4}\right) + 2 T_3\left(\frac{n}{16}\right) + 4n, m=2, k=1\\
            &\sum_{i=1}^{m} \alpha_i^k = \frac{3}{4} + 2\frac{1}{16} = \frac{7}{8} < 1 \Rightarrow \textrm{Fall 1!}\\
            &\Rightarrow T_3(n) = \Theta(n)
        \end{align*}
    \end{enumerate}


    \subsection*{Aufgabe 1.4}
    Für das kleinste $\alpha$ gilt: 
    \begin{align*}
        f(n) = 2\cdot f(n-1)+f(n-2) &\stackrel{!}{=} \alpha^n\\
        \Rightarrow 2\cdot \alpha^{n-1} + \alpha^{n-2} &= \alpha^n\\
        2 \alpha + 1 &= \alpha ^2 \\
        \alpha^2 - 2\alpha -1 &= 0 \\
        \Rightarrow \alpha_1 = 1 - \sqrt{2} < 0, \alpha_2 = 1 + \sqrt{2} > 0
    \end{align*}
    Mit den Bedingungen $f(1)=1, f(2)=2$ ergibt sich, dass Lösung $\alpha_2$ die richtige ist,
    um $f(n) \le \alpha^n$ zu erfüllen.


    \subsection*{Aufgabe 1.5}
    \begin{enumerate}
        \item[a)]
        Betrachten wir das Spiel mit den Stäben A (Start), B (Zwischenspeicher), C (Ziel) und einer Anzahl von Scheiben $n$. \\\\
        Im Fall $n=1$ ist die Aufgabe einfach: Die Scheibe wird von A nach C bewegt und das Spiel ist in einem Zug vorbei.\\\\
        Im Fall $n=2$ wird erstmals der Zwischenspeicher verwendet: Man bewegt zunächst die kleinste Scheibe von A nach B. Jetzt befindet
        man sich wieder in der Situation vom Anfang, in der eine einzelne Scheibe (die größere) von A nach C bewegt werden muss. Abschließend wird
        noch die kleine Scheibe von B nach C bewegt, und das Spiel ist nach 3 Zügen vorbei. \\\\
        Im Fall $n=3$ erkennt man, wie ein rekusives Muster ensteht: Die erste Aufgabe besteht darin, die Pyramide der obersten zwei Scheiben
        von A nach B zu bewegen. Dies geschieht in 3 Zügen, wie im Fall $n=2$. Anschließend kann die größte Scheibe einfach von A nach C bewegt werden und
        befindet sich damit an ihrem finalen Platz (+1 Zug). Nun besteht die Aufgabe erneut darin, die Pyramide mit zwei Scheiben von B nach C,
        unter Hilfe von A als Zwischenspeicher, zu bewegen. Dies geschieht erneut in 3 Zügen. Damit sind insgesamt 7 Zügen notwendig gewesen.\\\\
        Generell lässt sich das Problem also wie folgt formulieren: Die benötigte Anzahl an Zügen $B(n)$ für $n$ Scheiben entspricht der zweifachen Anzahl an
        Zügen bei $n-1$ Scheiben, plus 1, um die Scheibe selbst zu bewegen:
        \begin{align*}
            B(1) &= 1\\
            B(n) &= 2\cdot B(n-1) + 1
        \end{align*}
        Als Ansatz für die explizite Lösung wählen wir:
        \begin{align*}
            \textrm{I: } B(n)= 2\cdot B(n-1) + 1 &\stackrel{!}{=} a^n + c \\
            2\cdot B(n-1) + 1 &\stackrel{!}{=} a^n + c \\
            \textrm{mit I } \Rightarrow 2\left(a^{n-1}+c\right)+1 &\stackrel{!}{=} a^n + c \\
            2a^{-1}a^n + 2c + 1 &\stackrel{!}{=} a^n + c\\
            \textrm{Koeffizientenvergleich liefert:}\\
            2a^{-1}=1 &\Rightarrow a=2\\
            2c+1 = c &\Rightarrow c =1\\
            \Rightarrow B(n) &= 2^n-1
        \end{align*}
        \item[b)]
        Abgabe in DOMjudge. Teamname: "test" \\
        \textit{Die Aufgabe war verdammt schwer, zumindest in Anbetracht der Punktzahl! Ging das nur mir so?}
    \end{enumerate}

\end{document}