\documentclass{article}
\usepackage{amsmath}
\usepackage[utf8]{inputenc}
\usepackage[T1]{fontenc}
\usepackage[ngerman]{babel}
\usepackage[shortlabels]{enumitem}
\usepackage{amsfonts}
\usepackage[left=3cm,right=2cm,top=2.5cm,bottom=2cm]{geometry}
\usepackage{listings}
\usepackage{xcolor}

\definecolor{codegreen}{rgb}{0,0.6,0}
\definecolor{codegray}{rgb}{0.5,0.5,0.5}
\definecolor{codeorange}{rgb}{0.8,0.54,0.0}

\lstset{
  basicstyle=\footnotesize\tt,        % the size of the fonts that are used for the code
  breakatwhitespace=false,         % sets if automatic breaks should only happen at whitespace
  breaklines=true,                 % sets automatic line breaking
  captionpos=b,                    % sets the caption-position to bottom
  extendedchars=true,              % lets you use non-ASCII characters; for 8-bits encodings only, does not work with UTF-8
  language=Java,                 % the language of the code
  keywordstyle=\bf,
  showspaces=false,                % show spaces everywhere adding particular underscores; it overrides 'showstringspaces'
  showstringspaces=false,          % underline spaces within strings only
  showtabs=false,                  % show tabs within strings adding particular underscores
  tabsize=2,                   % sets default tabsize to 2 spaces
  commentstyle=\color{codegreen},
  keywordstyle=\color{magenta},
  numberstyle=\tiny\color{codegray},
  stringstyle=\color{codeorange},
}

\title{Einführung in die Informatik: Übung 2}
\author{Alexander Waldenmaier}

\begin{document}
    \maketitle
    \section*{Präsenzaufabe}
    \begin{itemize}
        \item \textit{Mathematisch}: $\mathrm{alter} <= 19$ \\
        \textit{In Java}: \begin{lstlisting}
        alter <= 19
        \end{lstlisting}
        \item \textit{Mathematisch}: $\neg\left(\mathrm{uhrzeit} >= t_{sonnenaufgang} \wedge \mathrm{uhrzeit} <= t_{sonnenuntergang}\right)$\\
        \textit{In Java}: \begin{lstlisting}
        !((uhrzeit >= t_sonnenaufgang) && (uhrzeit < t_sonnenuntergang))
        \end{lstlisting}
        \item \textit{Mathematisch}: $\neg \mathrm{istAngeschaltet}$\\
        \textit{In Java}: \begin{lstlisting}
        !istAngeschaltet
        \end{lstlisting}
        \item \textit{Mathematisch}: $(\mathrm{zucker} \wedge \neg \mathrm{milch}) \vee (\neg \mathrm{zucker} \wedge \mathrm{milch})$\\
        \textit{In Java}: \begin{lstlisting}
        zucker ^ milch
        \end{lstlisting}
    \end{itemize}


    \section*{Aufgabe 1: Pseudocode Algorithmus}
    \begin{lstlisting}[language=java, showspaces=false]
        int ware = get_rand_int(1, 5);
        int preis = get_rand_int(1, 10);
        int tries = 5;
        boolean success = false;
        while (tries > 0) {
            println("Ware eingeben (int zwischen 1 und 5)");
            in_ware = Input();
            println("Preis eingeben (int zwischen 1 und 10)");
            in_preis = Input();

            if ((ware == in_ware) && (preis == in_preis)) {
                success = true;
                break;
            } else {
                println("Sie haben falsch geraten!");
                if (in_ware > ware) {
                    println("Der Wert fuer Ware lag zu hoch!");
                } else if (in_ware < ware) {
                    println("Der Wert fuer Ware lag zu niedrig!");
                }
                if (in_preis > preis) {
                    println("Der Wert fuer Preis lag zu hoch!");
                } else if (in_preis < preis) {
                    println("Der Wert fuer Preis lag zu niedrig!");
                }
            }
            tries--;

        }
        if (success) {
            println("Sie haben erfolgreich geraten!");
        } else {
            println("Sie haben leider keine weiteren Versuche!");
            println("Die richtigen Werte waeren gewesen:");
            println("Ware = %d, Preis = $d", ware, preis);
        }


    \end{lstlisting}

    
    \section*{Aufgabe 2: Datentypen}
    \begin{itemize}
        \item Das "`kommt drauf an"' - generell ist bei der Zeitmessung wohl eher der Datentyp \textit{float}, vielleicht bei höheren Präzisionsanforderungen auch \textit{double} geeignet. Dies ist insbesondere der Fall, wenn beispielsweise die Systemuhr in der Einheit "`Sekunden"' ausgelesen wird und man beispielsweise Benchmarks zur Programmlaufzeit durchführt. Bei wissenschaftlichen Berechnungen ist die Verwendung von Kommazahlen selbstverständlich. Gibt die Uhr, mit der gemessen wurde, allerdings nur Ganzzahlen heraus (beispielsweise in der Einheit "`Mikrosekunden"') muss auch auf Programmseite etwa ein \textit{int} oder besser \textit{long} verwendet werden, um diese Eigenschaft wiederzuspiegeln. Damit ist dann auch klar, dass Bruchteile einer Mikrosekunde schlicht nicht erfasst werden können. 
        \item Hier verwendet man eindeutig den Datentyp \textit{char}.
        \item Hausnummern sind im Normalfall Ganzzahlen mit überschaubaren Größen, also am ehesten ein \textit{int} oder sogar \textit{short}. Allerdings gibt es auch Zusätze, z.B. Hausnummer "12a". Aus diesem Grund ist die Wahl eines \textit{char}-arrays (\textit{string}) wahrscheinlich besser.
        \item Je nach zu erwartender Anzahl an Nutzern funktionieren \textit{short} oder \textit{int}.
        \item Da es nur zwei mögliche Ausgänge gibt (wir nehmen an, dass die Münze nicht auf ihrer Kante landen kann), reicht sogar der Datentyp \textit{boolean}.
        \item Ein Bruch kann in Form zweier Zahlen vom Typ \textit{int} oder ähnlich wiedergegeben werden. Der Dezimalwert hingegen kann nur mit \textit{float} oder \textit{double} dargestellt werden (abgängig von der geforderten Präzision).
    \end{itemize}


    \section*{Aufgabe 3: Boolsche Ausdrücke}
    \begin{lstlisting}
          (A && !A) || !(5 != 6 ^ (1 > 42) == (23 < 23))
        = false || !(true ^ (false) == (false))
        = false || !(true ^ true)
        = false || !(false)
        = false || true
        = true
    \end{lstlisting}
\end{document}
