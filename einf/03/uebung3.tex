\documentclass{article}
\usepackage{amsmath}
\usepackage[utf8]{inputenc}
\usepackage[T1]{fontenc}
\usepackage[ngerman]{babel}
\usepackage[shortlabels]{enumitem}
\usepackage{amsfonts}
\usepackage[left=3cm,right=2cm,top=2.5cm,bottom=2cm]{geometry}
\usepackage{listings}
\usepackage{xcolor}
\usepackage{algorithm}
\usepackage[noend]{algpseudocode}

\definecolor{codegreen}{rgb}{0,0.6,0}
\definecolor{codegray}{rgb}{0.5,0.5,0.5}
\definecolor{codeorange}{rgb}{0.8,0.54,0.0}

\lstset{
  basicstyle=\footnotesize\tt,        % the size of the fonts that are used for the code
  breakatwhitespace=false,         % sets if automatic breaks should only happen at whitespace
  breaklines=true,                 % sets automatic line breaking
  captionpos=b,                    % sets the caption-position to bottom
  extendedchars=true,              % lets you use non-ASCII characters; for 8-bits encodings only, does not work with UTF-8
  language=Java,                 % the language of the code
  keywordstyle=\bf,
  showspaces=false,                % show spaces everywhere adding particular underscores; it overrides 'showstringspaces'
  showstringspaces=false,          % underline spaces within strings only
  showtabs=false,                  % show tabs within strings adding particular underscores
  tabsize=2,                   % sets default tabsize to 2 spaces
  commentstyle=\color{codegreen},
  keywordstyle=\color{magenta},
  numberstyle=\tiny\color{codegray},
  stringstyle=\color{codeorange},
}

\title{Einführung in die Informatik: Übung 3}
\author{Alexander Waldenmaier}

\begin{document}
    \maketitle

    \section*{Aufgabe 1: Teiler}
    \begin{algorithm}
        \begin{algorithmic}[1]
            \Procedure{berechneTeilerzahl}{start, end}
            \For {$n \gets start \textbf{ to } end$}
                \State $\textit{teiler} \gets 0$
                \For {$t \gets 2 \textbf{ to } n$}
                    \If {$n \text{ mod } t == 0$}
                        \State $\textit{teiler} \gets \textit{teiler}  + 1$
                    \EndIf
                \EndFor
                \State \textbf{print} \textit{teiler}
            \EndFor
            \EndProcedure
            \end{algorithmic}
    \end{algorithm}
   

    \section*{Aufgabe 2: Präzedenzregeln}
    \begin{lstlisting}
        42  +  5  *  -  2  !  =  4  %  2  +  2
        42  +  5  * (-2)   !=  4  %  2  +  2
        42  +  (5 * (-2))  !=  (4  %  2)  +  2
        (42 +  (5 * (-2))) !=  ((4  %  2)  +  2)
    \end{lstlisting}


    \section*{Aufgabe 3: Ausgaben}
    Die Ausgaben lauten:
    \begin{lstlisting}
        0.8999999999999999
        198!
        3
        Rechnung: 3-30
    \end{lstlisting}
    Erklärungen zu jeder Zeile: 
    \begin{enumerate}
        \item Da nur floats addiert werden, findet innerhalb des \textit{println} statements eine arithmetische Addition der drei Zahlen statt. Erst am Ende wird die resultierende Zahl in einen String verwandelt und dann ausgegeben. Der Grund warum nicht wie erwartet die $0.9$, sondern $0.8999999999999999$ ausgegeben wird, ist die endliche Genauigkeit, mit der floats gespeichert werden können. Bei der Verrechnung können daher kleine Ungenauigkeiten auftreten.
        \item Da die Buchstaben A, B und C als \textit{chars} deklariert wurden (erkennbar an der Verwendung der Hochkommas statt Anführungszeichen) werden diese in einen \textit{int} gemäß der ASCII-Konvention umgewandelt. Diese sind 65, 66 und 67. Die Summe dieser drei Zahlen ist 198. Da erst im Anschluss ein tatsächlicher \textit{string} folgt, findet erst dann eine Umwandlung des \textit{int} 198 in einen \textit{string} statt. Dieser wird dann um das Ausrufezeichen ergänzt.
        \item Da sowohl die 11 als auch die 3 ein \textit{int} sind, entsteht bei der Division ebenfalls eine Ganzzahl (Kommastellen werden abgeschnitten). Das Ergebnis ist der \textit{int} 3, der dann ausgegeben wird. 
        \item Da zu Beginn der \textit{string} \lstinline{"Rechnung  "} steht, werden die nachfolgenden "`Additionen"' als \textit{string}-Zusammenfügungen verstanden. Folglich werden die \textit{ints} 3, -3 und 0 jeweils in \textit{strings} umgewandelt und dann hintereinander ausgegeben. 
    \end{enumerate}

    
    \section*{Aufgabe 4: Grumpy cats}
    \begin{lstlisting}
        public class aufgabe4 {
            public static void main(String[] args){
                String prefix, suffix;
                for (int i = 2; i <= 100; i++) {
                    if (i % 5 == 0) {
                        prefix = "look! ";
                    } else {
                        prefix = "";
                    }
                    if (i % 3 == 0) {
                        suffix = " grumpy";
                    } else {
                        suffix = "";
                    }
                    System.out.println(prefix + i + suffix + " cats");
                }
            }
        }
    \end{lstlisting}


    \section*{Zusatzaufgabe (1/6): Primteiler}
    \begin{lstlisting}
        import java.util.ArrayList;

        public class zusatzaufgabe {
            public static void main(String[] args){
                // Zahl n von der Konsole einlesen
                Long n = Long.parseLong(args[0]);

                // Beliebig erweiterbare ArrayList verwenden um alle Faktoren zu sammeln
                ArrayList<Long> factors = new ArrayList<Long>(0);

                // Alle Teiler von 2 bis maximal sqrt(n) durchprobieren
                long t = 2;
                while (t * t < n) {
                    // Wenn t ein Teiler ist, diesen zur Liste hinzufuegen. Gleichzeitig n durch t teilen
                    if (n % t == 0) {
                        factors.add(t);
                        n /= t;
                    // Wenn kein Teiler, naechsthoehere Zahl probieren
                    } else {
                        t++;
                    }
                }
                // Das finale n ist eine Primzahl und wird auch zu den Faktoren ergaenzt
                factors.add(n);

                // Ausgabe erstellen durch Ergaenzung aller Primfaktoren zum Outputstring
                String output = "Primfaktorzerlegung von " + n + " = ";
                for (Long factor : factors) {
                    output += factor + "*";
                }
                output = output.substring(0, output.length()-1);

                // Finale Ausgabe
                System.out.println(output);
            }
        }
    \end{lstlisting}
\end{document}
