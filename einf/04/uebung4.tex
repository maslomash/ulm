\documentclass{article}
\usepackage{amsmath}
\usepackage[utf8]{inputenc}
\usepackage[T1]{fontenc}
\usepackage[ngerman]{babel}
\usepackage[shortlabels]{enumitem}
\usepackage{amsfonts}
\usepackage[left=3cm,right=2cm,top=2.5cm,bottom=2cm]{geometry}
\usepackage{listings}
\usepackage{xcolor}
\usepackage{algorithm}
\usepackage[noend]{algpseudocode}

\definecolor{darkviolet}{rgb}{0.5,0,0.4}
\definecolor{darkgreen}{rgb}{0,0.4,0.2} 
\definecolor{darkblue}{rgb}{0.1,0.1,0.9}
\definecolor{darkgrey}{rgb}{0.5,0.5,0.5}
\definecolor{lightblue}{rgb}{0.4,0.4,1}

\lstset{
    language=Java,
    basicstyle=\small\ttfamily,
    keywordstyle=\color{darkviolet}\bfseries,
    commentstyle=\color{darkgreen},
    stringstyle=\color{darkblue},
    morecomment=[s][\color{lightblue}]{/**}{*/},
    showstringspaces=false,
    numbers=right
}

\title{Einführung in die Informatik: Übung 4}
\author{Alexander Waldenmaier}

\begin{document}
    \maketitle

    \section*{Präsenzaufgabe}
    \begin{enumerate}
        \item[a)] 
        \begin{lstlisting}
            import java.util.Scanner;

            public class MyLoop{
                public static void main(String[] args) {
                    Scanner scan = new Scanner(System.in);
                    System.out.print("Insert a number: ");
                    int k = scan.nextInt();

                    int i = 1; 
                    do {
                        System.out.println(i*i);
                        i++;
                    } while i < (k-1);
                }
            }
        \end{lstlisting} \hfill\\
        \item[b)] 
        \begin{lstlisting}
            int k = ...;
            int x = 0;

            for (i=0; i<k;) {
                x += k * ++i;
            }
            System.out.println("x: " + x);
        \end{lstlisting}\hfill \\
        \item[c)] 
        \begin{lstlisting}
            int k = ...;
            if (k>0) {
                int m = 0;
                for (i=1, i<k, i++) {
                    if (k*i > m)
                        m = k + i;
                }
                System.out.println("m: " + m);
            } 
        \end{lstlisting}
    \end{enumerate}


    \subsection*{Aufgabe 1: Schwobifying Light}
    Siehe Abgabe "`aufgabe1.java"'.


    \subsection*{Aufgabe 2: Zufallsmuster}
    Siehe Abgabe "`aufgabe2.java"'.\\\\
    Der Algorithmus terminiert immer. Theoretisch wäre es möglich, dass unendlich oft der Zufall eine 0 ergibt und nie eine 1, dennoch aber würde nach jeder 0 geprüft werden, ob bereits $n$ Nullen erreicht sind. Wenn ja, werden alle verfügbaren Nullen ausgegeben und der Algorithmus beendet sich.


    \subsection*{Aufgabe 3: Verschlüsselung}
    \begin{enumerate}
        \item[a)] Siehe Abgabe "`Decrypt.java"'. Die verschlüsselten Sätze (und zugehörige Rotation) lauten:
        \begin{itemize}
            \item oppa gangnam style (ROT 81)
            \item What does the fox say? (ROT 77)
            \item Tuebingen warum bist du so huegelig? (ROT 89)
            \item MS-Dos Manfred und Loetkolben Ludwig (ROT 91)
        \end{itemize}
        \item[b)] Siehe Abgabe "`Decrypt2.java"'. Die Ergebnisse lauten:
        \begin{itemize}
            \item Fall 1:
            \begin{lstlisting}[numbers=none]
                64: <`eu<j\cu\iek\kue\leu<i[Y\\i\e
                73: Ein Esel erntet neun Erdbeeren
            \end{lstlisting}
            \item Fall 2 (mit der vorgegebenen Wahrscheinlichkeit 10\%):
            \begin{lstlisting}[numbers=none]
                58: +HZe/\OUeMYPZ[eRLPUe.YHZ
            \end{lstlisting}
            Wählt man eine geringere Wahrscheinlichkeit, zum Beispiel 1\%, ergibt sich folgendes: 
            \begin{lstlisting}[numbers=none, language=bash]
                18: a 2=e4'-=%1(23=*$(-=d1 2
                19: b!3>f5(.>&2)34>+%).>e2!3
                22: e$6Ai8+1A)5,67A.(,1Ah5$6
                58: +HZe/\OUeMYPZ[eRLPUe.YHZ
                67: 4Qcn8eX^nVbYcdn[UY^n7bQc
                68: 5Rdo9fY_oWcZdeo\VZ_o8cRd
                69: 6Sep:gZ`pXd[efp]W[`p9dSe
                70: 7Tfq;h[aqYe\fgq^X\aq:eTf
                74: ;Xju?l_eu]i`jkub\`eu>iXj
                77: >[mxBobhx`lcmnxe_chxAl[m
                79: @]ozDqdjzbneopzgaejzCn]o
                80: A^p{Erek{cofpq{hbfk{Do^p
                82: C`r}Gtgm}eqhrs}jdhm}Fq`r
                83: Das Huhn frist kein Gras
                87: Hew$Lylr$jvmwx$oimr$Kvew
            \end{lstlisting}    
        \end{itemize} 
        Bei einer Verschiebung um 83 Zeichen ergibt sich bei Fall 2 tatsächlich der Satz "Das Huhn frist kein Gras". Dabei wird auch der Nachteil dieses Verfahrens ersichtlich: Bei kurzen Sätzen oder gar einfachen Wörtern gelingt es selten, den tatsächlichen Durchschnitt von 17,4 \% "`e"' zu erreichen. In diesen Fällen muss die erforderliche Prozentzahl deutlich niedriger gewählt werden. 
    \end{enumerate}

\end{document}
