\documentclass{article}
\usepackage{amsmath}
\usepackage[utf8]{inputenc}
\usepackage[T1]{fontenc}
\usepackage[ngerman]{babel}
\usepackage[shortlabels]{enumitem}

\title{Einführung in die Informatik: Übung 1}
\author{Alexander Waldenmaier}

\begin{document}
    \maketitle
    \section*{Pr"asenzaufgabe}
    Als Input f"ur den Algorithmus k"onnte eine Liste mit $n$ Zahlen dienen, wobei das i-te Listenelement als $a_i$ bezeichnet
    wird (mit $0\le i<n$). Der Algorithmus könnte die folgenden Schritte beinhalten:
    \begin{enumerate}[(a)]
        \item Setze $i=0$ (1E)
        \item Teste $i<(n-1)$ (1E)
        \item Gebe aus "nein, $a_i$ ist die letzte Zahl!" (1E)
        \item Setze $j=i+1$, Teste $a_i < a_{j}$ (2E)
        \item Gebe aus "ja, $a_j$ ist größer!", Setze $i=i+1$ (2E)
        \item Setze $j=j+1$, Teste $j<n$ (2E)
        \item Gebe aus "nein, es gibt keine größere Zahl!", Setze $i=i+1$ (2E)
    \end{enumerate}

    Zu Beginn des Programmdurchlaufs wird in Schritt a die Laufvariable $i=0$ gesetzt. 
    In Schritt b wird geprüft, ob bereits das letzte Element der Liste erreicht ist. 
    Wenn ja, dann wird in Schritt c eine entsprechende Meldung herausgegeben und das Programm terminiert.
    Wenn nicht, wird in Schritt d die Laufvariable $j=i+1$ gesetzt und geprüft, ob das j-te Element größer
    als da i-te Element ist. 
    Wenn ja, wird in Schritt e eine entsprechende Meldung herausgegeben, $i$ inkrementiert und
    wieder zu Schritt b zurückgekehrt. 
    Wenn nicht, wird in Schritt f $j$ inkrementiert und geprüft, ob damit das Ende der Liste erreicht ist. 
    Wenn ja, wird in Schritt g eine entsprechende Meldung herausgegeben, $i$ inkrementiert und wieder zu Schritt
    b zurückgekehrt. Wenn nicht, wird direkt zu Schritt d zurückgekehrt. 

    Im besten Fall handelt es sich um eine Liste mit $n$ Zahlen, bei denen jede Zahl gr"o"ser als ihr Vorg"anger ist (oder die erste ist).
    Dann liegt die minmale Ausf"uhrungszeit vor, da f"ur jedes Listenelement $X$ nur das n"achste Element $Y$ "uberpr"uft werden muss
    und dabei stets sofort $Y>X$ gilt. 

    \begin{align*}
        T_{min} &= 1*a + n*b + 1*c + (n-1)*d + (n-1)*e + (n-1)*f \\
        &= (1 + n + 1 + 2(n-1) + 2(n-1) + 2(n-1))E \\
        &= (7n-4)E
    \end{align*}

    Im schlechtesten Fall ist die Liste aber in fallender Reihenfolge sortiert. Dann muss f"ur jedes Listenelement $X$ jedes 
    darauffolgende Element bis an das Ende der Liste geprüft werden. Dann gilt:
    \begin{align*}
        T_{max} &= 1*a + n*b + 1*c + n\frac{n-1}{2}*d + (n-1)\frac{n-2}{2}*f + (n-1)*g \\
        &= (1 + n + 1 + 2\frac{n^2-n}{2} + 2\frac{n^2-3n+2}{2} + 2(n-1))E\\
        &= (2n^2 - n + 2)E
    \end{align*}

    Im Fall von $n=5$ gilt also $T_{min}=31E$ und $T_{max}=47E$. Aufgrund des quadratischen Wachstums von $T_{max}$ ist dieser
    Algorithmus bei hohen $n$ nicht praktikabel. 



    \section*{Aufgabe 1: Horner-Schema}
    Umrechnungen:
    \begin{itemize}
        \item $1000010_{(2)} = (((((1*2+0)*2+0)*2+0)*2+0)*2+1)*2+0 = 66_{(10)}$ \\
        \item $10222_{(8)} = (((1*8+0)*8+2)*8+2)*8+2 = 4.242_{(10)}$ \\
        \item $ABBA_{(16)} = ((10*16+11)*16+11)*16+10 = 43.962_{(10)}$
    \end{itemize}
    Schema rückwärts:
    \begin{equation*}
        11_{(10)} = 1*2^3+0*2^2+1^*2^1+1*2^0 = 1011_{(2)}
    \end{equation*}



    \section*{Aufgabe 2: Algorithmenkonstruktion}
    \begin{quotation}
        Finde die minimale, maximale und durchschnittliche Abarbeitungszeit! 
    \end{quotation}

    \subsubsection*{a) Problemspezifikation}
    Was ist die Abarbeitungszeit? Die Abarbeitungszeit ist auf Programmniveau ein Zahlenwert aus der gegebenen
    Menge. Folglich sind "`minimal"', "`maximal"' und "`durschnittlich"' nun als mathematische Begriffe zu verstehen, 
    die auf die in der Menge befindlichen Zahlenwerte anzuwenden sind. 

    \subsubsection*{b) Problemabstraktion}
    \begin{itemize}
        \item Gegeben: Eine Menge $X$ von positiven Float-Zahlen. Die Menge enthält $n$ Elemente, jedes Element sei $x_i$ genannt (wobei $0\le i <n$)
        \item Gesucht: Das Minimum $min(X)$, das Maximum $max(X)$ sowie der Durschnitt $sum(X)/n$ der Arbarbeitungszeiten.
    \end{itemize}


    \subsubsection*{c) Algorithmenentwurf}
    \paragraph*{Algorithmus "`Abarbeitungszeiten"':}
    \begin{enumerate}[(a)]
        \item Lese die Menge $X$ als Liste von Float-Zahlen $x_i$ mit Länge $n$ ein. (2E)
        \item Setze die Variablen $x_{min}=x_0$, $x_{max}=x_0$, $x_{sum}=0$ und $i=0$ (4E)
        \item Solange $i<n$, wiederhole: (1E)
        \begin{enumerate}
            \item Wenn $x_i$ < $x_{min}$ (1E)
            \begin{enumerate}
                \item Setze $x_{min} = x_i$ (1E)
            \end{enumerate}
            \item Wenn $x_i$ > $x_{max}$ (1E)
            \begin{enumerate}
                \item Setze $x_{max} = x_i$ (1E)
            \end{enumerate} 
            \item Setze $x_{sum} = x_{sum} + x_i$ (1E)
            \item Setze $x_i = x_i + 1$ (1E)
        \end{enumerate}
        \item Gebe aus:
        \begin{enumerate}
            \item "Minimale Ausführungsdauer = $x_{min}$" (1E)
            \item "Maximale Ausführungsdauer = $x_{max}$" (1E)
            \item "Durschnittliche Ausführungsdauer = $x_{sum}/n$" (1E)
        \end{enumerate}
    \end{enumerate}


    \subsubsection*{d) Korrektheitsnachweis, Verifikation}
    \paragraph*{Behauptung:}
    Der Algorithmus "`Abarbeitungszeiten"' terminiert und liefert die geforderten Ergebnisse.
    \paragraph*{Beweis:}
    Vollständige Induktion über $n$:
    \begin{itemize}
        \item Induktionsbehauptung: \\
        $x_{min}$ ist das Minimum von $X$, $x_{max}$ ist das Maximum von $X$ und $x_{sum}/n$ ist der Durchschnitt von $X$. Das Programm termininiert nach $n-1$ Durchläufen.
        \item Induktionsanfang ($n=1$): \\
        Beweis durch Simulation (Schritte wie in Teil c)): 
        \begin{enumerate}[(a)]
            \item Es wird gesetzt: $X={x_0}$, $n=1$
            \item Es wird gesetzt: $x_{min}=x_0$, $x_{max}=x_0$, $x_{sum}=0$ und $i=0$
            \item Schleife wird einmalig ausgeführt. 
            \item Es wird ausgegeben: 
            \begin{enumerate}
                \item "Minimale Ausführungsdauer = $x_{min}=x_0$"
                \item "Maximale Ausführungsdauer = $x_{max}=x_0$"
                \item "Durschnittliche Ausführungsdauer = $x_{sum}/n=x_0$"
            \end{enumerate}
        \end{enumerate} \\
            $\Rightarrow$ Die Behauptung gilt für $n=1$.
        \item Induktionsschritt ($n \rightarrow n+1$): \\
        Induktionssannahme: Induktionsbehauptung gelte für $n$, betrachte nun $n' = n+1$ (Schritte wie in Teil c)):\\
        \begin{enumerate}[(a)]
            \item Es wird gesetzt: $X$, $n=n'$
            \item Es wird gesetzt: $x_{min}=x_0$, $x_{max}=x_0$, $x_{sum}=0$ und $i=0$
            \item Schleife wird $n-1$ mal ausgeführt. Dann gilt nach Induktionsannahme die Induktionsbehauptung. \\
            Bei der letzten Ausführung der Schleife gibt es folgende Abwägungen: 
            \begin{enumerate}
                \item Wenn $x_{n+1}$ kleiner als $x_{min}$ ist, wird $x_{min} = x_{n+1}$ gesetzt. 
                \item Wenn $x_{n+1}$ größer als $x_{max}$, wird $x_{max} = x_{n+1}$ gesetzt. 
                \item Es wird $x_{sum} = x_{sum} + x_{n+1}$ gesetzt.
            \end{enumerate}\\
                In jedem Fall gilt die Induktionsbehauptung für $n'$. 
            \item Es wird ausgegeben: 
            \begin{enumerate}
                \item "Minimale Ausführungsdauer = $x_{min}$"
                \item "Maximale Ausführungsdauer = $x_{max}$"
                \item "Durschnittliche Ausführungsdauer = $x_{sum}/n$"
            \end{enumerate}
        \end{enumerate} \\
        
    \end{itemize}
    \textbf{q.e.d.}


    \subsubsection*{e) Aufwandsanalyse}
    Die Schritte a und b werden immer exakt einmal ausgeführt. Anschließend wird Schritt c $n-1$ mal ausgeführt. 
    Schritt c-a wird ebenfalls $n-1$ mal ausgeführt, c-a-i hingegen nur so häufig, wie $x_i$ kleiner als $x_{min}$ ist.
    Schritt c-b wird analog $n-1$ mal ausgeführt, c-b-i hingegen nur so häufig, wie $x_i$ größer als $x_{max}$ ist.
    Anschließend werden immer, insgesamt $n-1$ mal, die Schritte c-c und c-d ausgeführt.
    Schritte d-a, d-b und d-c werden jeweils exakt einmal ausgeführt. \\\\
    Die minimale Ausführungsdauer ergibt sich für den Fall, dass alle Elemente $x_i$ aus $X$ exakt gleich sind. In diesem Fall schlägt
    weder Test c-a noch Test c-b an und folglich werden c-a-i und c-b-i nie ausgeführt (alle Elemente sind sowohl das größte als auch 
    das kleinste Element). Im schlimmsten Fall sind alle Elemente unterschiedlich und ihrer Größe nach sortiert. Dann schlägt bei absteigender
    Reihenfolge jedes Mal Test c-a an und c-a-i wird ausgeführt. Bei umgekehrter Sortierung wird stets Test c-b anschlagen und c-b-i ausgeführt.

    \begin{align*}
        T_{min} = &1*a + 1*b +(n-1)*c + \\ &(n-1)*([c-a] + [c-b] + [c-c] + [c-d]) + 1 * d\\
        = &(1*2 + 1*4 + (n-1)*1 + (n-1)*(1 + 1 + 1 + 1) + 1*3)E\\
        = &(5n+4)E
    \end{align*}
    

    \begin{align*}
        T_{max} = &1*a + 1*b +(n-1)*c + \\ &(n-1)*([c-a] + [c-b] + [c-a/b-i] + [c-c] + [c-d]) + 1 * d\\
        = &(1*2 + 1*4 + (n-1)*1 + (n-1)*(1 + 1 + 1 + 1 + 1) + 1*3)E\\
        = &(6n+3)E
    \end{align*}


    \subsubsection*{Abschließende Bemerkungen}
    Wichtig für die korrekte Funktion des Algorithmus ist, dass sich in der Menge $X$ der Abarbeitungszeiten sämtliche Szenarien wiederfinden, 
    also insbesondere die "`Extremfälle"' bei denen Abarbeitung besonders lange gedauert hat, oder sehr schnell vonstatten ging. Wurden zufällig nur "gute" Abarbeitungszeiten erwischt, wird der Algorithmus auch nur diese
    Situation wiedergeben. 
\end{document}